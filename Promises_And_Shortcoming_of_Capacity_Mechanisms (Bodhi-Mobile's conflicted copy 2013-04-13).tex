\documentclass[11pt,a4paper,english]{scrreprt}
\usepackage[T1]{fontenc}
\usepackage{lmodern}
\usepackage[ansinew]{inputenc}


\usepackage{xcolor}
\usepackage{relsize}
\usepackage{paralist}
\usepackage{typearea}
\usepackage{setspace}
\usepackage{textcomp}
\usepackage{marvosym}
\usepackage{amsmath}
\usepackage{array}
\usepackage{booktabs}


%%%----------------------------------------------------------------------
\definecolor{dunkelgrau}{gray}{0.20}
\definecolor{hellgrau}{gray}{0.40}

\onehalfspacing
\typearea[current]{calc}

\newcommand{\changefont}[3]{
\fontfamily{#1} \fontseries{#2} \fontshape{#3} \selectfont}

\newcommand{\textemph}{\textsl{\textbf}}
%%%----------------------------------------------------------------------
\begin{document}

\begin{spacing}{1}

	\begin{titlepage}
		
		\begin{center}
		{\large Freie Universit\"{a}t Berlin\\
		Otto-Suhr\texttwelveudash
		Institut f\"{u}r Politikwissenschaft\\
		HS #####\\
		\emph{TITLE OF TERM PAPER}\\
		\textsmaller{\textbf{Lecturer:} LECTURER''}}\par
		
		\vspace{4cm}
		
\changefont{ppl}{b}{n}
\sffamily{
  {\Huge{\textsmaller{\textbf{Term Paper}}}\\
 \bigskip	 
 \textcolor{dunkelgrau}{The Need For Capacity Mechanisms In Liberalized
Electricity Markets And Their Limits}}}
		
\par

\vspace{4cm}
		
\changefont{lmr}{m}{n}
		
		{\large Pascal Bernhard\\
		Schwalbacher Stra{\ss}e 7\\
		12161 Berlin\\
		Matrikelnummer: 3753179\\
		pascal.bernhard@belug.de}
		
		
		
		\end{center}

	\end{titlepage}
	
\end{spacing}

%%%----------------------------------------------------------------------


%%%----------------------------------------------------------------------
\tableofcontents


%%%----------------------------------------------------------------------
\setlength{\parindent}{30pt}
\setlength{\parskip}{0pt}


%%%-----------------------------------------------------------------------

\chapter{Introduction}

As many developed countries around the world have been liberalizing their
wholesale electricity markets to different degrees during the last three
decades\footnote{Chile and UK were the first countries to abolish generation
monopolies and opening up their markets to private companies introducing
supply-side competition. Several US states and EU countries followed in the
1990s and 2000s}, their governments, regulators and customers had to make the
experience that not all the hoped-for benefits of these steps did
materialize\footnote{see IEA (2002, 2003)}. Economic theory had predicted that
competitive markets should lead to cheaper electricity as prices would
eventually converge towards marginal production costs. The market mechanism was
thought to provide the necessary price signals to give generators enough
incentive to invest in new capacity so that supply and demand could be matched
in the future\footnote{Finon \& Pignon (2006), p.3}.\par

Alas, market designers didn't pay enough attention to intrinsic imperfections
of electricity markets leading to insufficient investments in generation
capacity. Inelasticities on the supply as well as the demand side and the
difficulty to store larger amounts of electricity for later use result in high
price volatility on the wholesale market\footnote{ibid (2006), p.9}.
Together with regulators' unwillingness to allow spot prices to climb too high
fearing that firms exercise market power, new investments became unattractive to
power companies thus leading to what became called the 'missing-money
problem'\footnote{The 'missing-money problem' refers to the issue that so-called
peak power plants which are in use only for a small part of the year (during
peak demand periods) need high prices to cover their considerable fixed costs.
If there are price caps in place, these prices will not come about which in turn
results in under-investment in capacity.}.\par

This issue particularly concerns so-called peak power
plants\footnote{B\"{o}ckers, Giesing, Haucap, Heimeshoff \& R\"{o}sch (2012a),
p.4} whose short-term response times to sudden increases in demand make them
indispensable for the system operator to ensure system reliability avoiding
involuntary rationing by shedding load.\par

Aggravating the situation is the increasing share of intermittent renewable
sources especially in the German electricity market. Spot prices experience a
downward pressure as renewables feature marginal production costs near zero and
their bids on the wholesale market have to be taken according to German
legislation\footnote{Diermann, Carsten, von Hammerstein, Christian, Hermann,
Hauke Matthes, Felix Chr. \& Schlemmermeier, Ben (2012), p.36}. This development
makes it even more difficult for peak power plants to recoup investment costs.
As a result from the year 2020 on, several authors predict a lack of peak
capacity in Germany\footnote{ibid (2012), p.25 also Grave, Lindenberger \&
Paulus (2012)}.\par

Capacity adequacy, defined as supply security in the long-run, can arguably be
considered a public good, being non-rival but also non-excludable\footnote{Finon
\& Pignon (2006), p.3}. System failure caused by insufficient generation imposes
heavy losses on all market participants be they consumers or suppliers of
electric power. Having the characteristics of a public good, private actors will
not provide system reliability unless incentivized or forced to. Government
intervention could be a solution to this problem of capacity shortages.\par


%%%%%%%%%%%%%%%%%%%%%%%%%%%%%%%%%%%%%%%%%%%%%%%%%%%%%%%%%%%%%%%%%%%%%%%%%%%%%%%%

\chapter{Structure \& Research Question}
	

This paper looks into the issue whether capacity instruments\footnote{This paper
will employ the terms \emph{capacity instrument}, \emph{capacity mechanism} and
\emph{capacity market} interchangeably. Being linguistically accurate the
expression \emph{capacity market} should refer to the entire market system,
whereas \emph{capacity instruments} are the individual tools interacting with
each other via \emph{capacity mechanisms} to constitute the \emph{capacity
market}. For reasons of readibility I will use the three terms analogously.} can
solve the aforementioned 'missing-money problem' in the long-term. This has to
remain largely a theoretical discussion as the results and unintended effects of
market liberalization\footnote{The term liberalization refers here to reforms to
national electricity markets that attempted to introduce competition. Measures
taken included vertical unbundling by which incumbent energy firms had to
separate different business activities along the value chain (generation,
transmission, distribution to end-consumers), granting access to monopoly
network infrastructure to third parties and the creation of independent
regulators. Liberalization often went along with full or partial privatization
of state-owned electricity assets. Market liberalization has been implemented to
varying degrees in different countries, and this paper will take the
developments on Germany's electricity market as an illustration. Nonetheless the
conclusion drawn should be applicable to a certain extent to any deregulated
electricity market} have become fully visible only during the past decade in
most countries\footnote{see IEA (2002)}. The idea of capacity mechanisms a
fairly new approach to the problem and in many markets hasn't even been tried
yet. Thus empirical data are rather thin and Germany which is to provide some
exemplary illustrations for the questions at hand does not have any capacity
market to date. On the other hand repeated interruptions in the wake of
California's electricity market reforms in the 1990s\footnote{see Jurewitz
(2002) as well as Woo (2001)} show that concern about how to guarantee future
generation adequacy is not merely a theoretical thought experiment.\par


In Germany and other countries the generation parc has not run through a
complete investment cycle since market deregulation\footnote{The terms
\emph{deregulation} and \emph{liberalization} will be used synonymously here,
although deregulation, the removal of sector specific regulation, can be
considered a subset of a more complex liberalization effort (see Sioshansi
2006a)} in the late 1990s. So empirically no final judgment can be made
whether 'energy-only'--markets\footnote{On an \emph{'energy-only'--market}
solely electrical power is traded, generation and transmission capacity are not
part of the transactions} provide sufficient incentives to build new plants or
not.\par


	\paragraph{Proceeding}

After an overview of the specific characteristics of electricity and
electricity markets the paper will describe the 'missing-money problem' and why
liberalized 'energy-only'--markets may not be able to guarantee sufficient
investment in capacity for ensuring adequate supply. In the third part I will
present the challenges to the established 'merit-order' from the massive
increase in renewable electricity generation and how this aggravates the
'missing-money problem'.\par


My paper argues that capacity mechanisms will fail to address the deficiencies
of 'energy-only'--markets if regulators inhibit the free functioning of markets
by maintaining caps on spot prices. The attempt to curb the unwelcome effects of
market liberalization together with the expansion of renewable energy's market
share will make it more difficult to ensure supply adequacy on the one hand. At
the other hand this will drastically limit the potential welfare gains of a
deregulated market.\par


This paper does not have the scope to explore advantages and disadvantages of
different capacity schemes, as there are many, but will highlight general
shortcomings of this solution. For a thorough discussion on the effectiveness of
specific designs see Battle \& Per\'{e}z-Arriaga (2008), Diermann et al. (2012)
and Siegmeier (2012).\par




	

	
%%%%%%%%%%%%%%%%%%%%%%%%%%%%%%%%%%%%%%%%%%%%%%%%%%%%%%%%%%%%%%%%%%%%%%%%%%%%%%%%


\chapter{The Need For Capacity Mechanisms}	
	

  \section{Electricity As A Good Unlike Others}
  
   
Electric power is unlike other goods or commodities in that it is expensive
to store for later use. Given the current technology level only pump storage
plants allow this on a large scale, geographic conditions permitting, which
limits the feasibility of the approach to mountainous areas. Electricity
markets thus require continuous and above all instantaneous balancing between
supply and demand\footnote{Creti \& Fabra (2007), p.259/260}, as there is only
a very limited amount of stored power available to make up for generation
shortfalls.\par


Failure to balance supply and demand at any given time will put the system's
stability at risk. This can result in disruptions for the entire
electricity network due to involuntary load shedding as customers are
forcibly cut off from power supplies by the system operator. As only in rare
instances it is technically possible to suspend deliveries to specific
end-consumers their willingness to pay higher prices to continue the service or
else forgo supply cannot be acknowledged by the market.\par


	\paragraph{Inelasticities On Both Sides}

In addition, most households have fixed-price contracts with utilities making
them unresponsive to price spikes caused by supply scarcity\footnote{Hughes \&
Parece (2002), p.32}. This renders the demand side very inelastic in the short-
as well as the long-run\footnote{Marty (2007), p.274}.  To introduce demand
response to supply scarcity would be the installation of real-time meters which
transmitted price signals from the wholesale market to customers\footnote{ibid
(2007), p.273} would be one option. As neither utilities nor consumers are keen
on paying for this solution, the idea has not made much progress. Thus the
'traditional' function of market mechanisms to restore the equilibrium between
demand and supply by price swings as classic economics would have it cannot be
provided by the current market design.\par
  
  
On the supply side we also find short-term inelasticities since lead-times for
the construction of new power plants are considerable, ranging from between 2-3
years for gas-fired installations to 10 years or more for nuclear
reactors\footnote{Flinkerbusch \& Scheffer (2013), p.15}. The lump-sum nature of
investments in generation capacity with hefty upfront costs for new
infrastructure also leads to slow decision-making processes by power companies.
By implication the system operator has to estimate future demand for specific
time frames and then make sure that sufficient capacity will be available during
that period.\par
  
  
If capacity supply is inflexible and highly used at a certain point in time,
even small changes in available capacity or demand can result in huge price
swings\footnote{Hughes \& Parece (2002), p.32}. This inherent price volatility
in the electricity sector is unwelcome to generation companies since it
complicates estimating future revenues from new capacity very tricky thus making
these investments less likely to be undertaken.\par


    \paragraph{System Stability \textemdash{} A Public Good}
  
While scarcity of supply makes the available generation capacity very valuable
when wholesale prices are high, it renders it utterly valueless in case the
whole system collapses and no electricity can be sold anymore\footnote{Joskow
\& Tirole (2007), p.63}. The mere possibility of extreme cases where the system
breaks down under too much load or when the system operator resorts to
involuntary load shedding to prevent this happening, gives capacity adequacy the
character of a public good. No customer can be excluded from the benefits of a
functioning electricity system since it is nearly always impossible to cut off
individual consumers. By implication providing the good 'supply security' is not
a profitable business in itself and will thus not be furnished by an
'energy-only'--market.\par
  

Joskow and Tirole deem regulatory action all the more warranted as a single
power plant not supplying the amount of electricity it was contracted for could
potentially cause system failure. This would impose a severe negative
externality on all market participants\footnote{ibid (2007),
p.78} since they would all be affected immediately.\par
  
  
  
  \section{Electric Power Markets And Their Specific Nature}

To serve peak loads at any time there needs to be idle reserve capacity
available, which does not earn any income when not supplying electricity to
the system\footnote{Hughes \& Parece (2002), p.33}. Therefore, peak power plants
have to make high returns when called upon in order to recover not only
marginal costs but also their long-term fixed costs which only high spot prices
on the wholesale market could ensure. Operators that provide peak capacity will
not bid their marginal costs as in that case the assets would not be
profitable\footnote{Ni, Wen \& Wu (2004), p.366} without additional capacity
payments to fund their investment expenses. 

At the same time most electricity markets feature caps for wholesale prices
imposed by regulators. Market supervisors fear that high prices are less a
signal of scarce supply, but rather the result of strategic behavior by energy
firms using their market power. By withholding some of their capacity,
artificial scarcity can be induced which would in turn lead to higher prices if
the price mechanism was totally flexible and free. 

As policy makers and regulators resort to price limits on electricity, the
chances of peak power plants becoming sufficiently profitable to give
incentives for new investment\footnote{Keller \& Wild (2004):
\textsl{\textquotedblleft Investment only makes sense (i.e. is profitable) when
the discounted value of revenues from sales of new [\dots] capacity exceeds
investment and operation costs\textquotedblright}, p.244} are significantly
reduced. Thus peak power plants, although essential to ensure adequate
generation capacity during periods of high demand, pose substantial investment
risks for private actors\footnote{Finon \& Pignon (2006), p.4}. Adding to that
problematic situation, construction of new conventional power plants is rarely
feasible in incremental small-scale steps. Lump-sum investments imply higher
risks and by easing undersupply via extra capacity may paradoxically undercut
their initial economic rationale\footnote{Keller \& Wild (2004), p.244}.\par




%%%%%%%%%%%%%%%%%%%%%%%%%%%%%%%%%%%%%%%%%%%%%%%%%%%%%%%%%%%%%%%%%%%%%%%%%%%%%%%%


\chapter{The Effects Of Renewable Energy Sources}

The existing system of selecting supply offers according to the 
merit-order\footnote{\emph{Merit-Order:} Sorting of power plant types according
to their marginal generation costs in increasing order} gets disrupted as
renewable energy sources (photovoltaics, wind \& tidal power) increase their
market share. These technologies combine high fixed-costs (in relation to the
specific generation potential for an individual plant) with the advantage of
very low marginal costs since their input factors are provided free by
nature.\par

Bidding with very low offers renewable energy pushes all bids by other generation
technology rightwards in the merit-order. Thus the previously most expensive
conventional plants (regarding marginal costs) are not contracted for by the
system operator anymore, leaving this capacity unused and thus unprofitable. In
general, spot prices, reflecting only marginal prices, are pushed lower during
non-peak times. All participating plants are affected by this structural shift
in the wholesale price level, since all receive the same unique amount
established through the merit-order. These unique spot prices are insufficient
though for expensive conventional power plants to recover their fixed costs via
the difference between their marginal costs and spot prices. With these lower
returns conventional capacity thus finds it more difficult to pay off investment
expenses. Some plants will be even rendered entirely unprofitable since they get
called upon too rarely\footnote{B\"{o}ckers, Giesing, Haucap, Heimeshoff \&
R\"{o}sch (2012b) p.10}.

 
 %%%%%%%%%%%%%%%%%%%%%%%%%%%%%%%%%%%%%%%%%%%%%%%%%%%%%%%%%%%%%%%%%%%%%%%%%%%%%%%

 
\chapter{Shortcomings Of Capacity Instruments}



As this paper mentioned in its introductory part, we do not have extensive
empirical data so far to make a thorough assessment of how successful capacity
markets really are in practice. In most countries capacity instruments have not
been in place for long enough to judge their performance over the entire period
of an investment cycle in the electricity sector (20-30 years)\footnote{see
Diermann, von Hammerstein, Hermann, Matthes \& Schlemmermeier (2012)}. The
original 'energy-only'--market structures and the solutions politicians in
different countries have found to tackle the 'missing-money problem' are so
diverse that a \textsl{most-similiar} or \textsl{most-dissimiliar} empirical
research design would be very difficult and probably unreliable to apply.\par


Therefore I will limit the following discussion on capacity markets to a
theoretical level, which should yield informative insights nonetheless. Even
with the implementation of a certain capacity mechanism additional objectives
policy makers may want the national electricity market to achieve cannot easily
be squared with the aim to ensure future electricity supply. Low prices for
industry to maintain international competitiveness and for households out of
consideration for social policy purposes will barely provide the price signals
we have seen are necessary to make investment in new capacity attractive.\par


Several important aspects of supply adequacy will be left aside due to the
limits of scope of this term paper. The effects of an increasingly integrated
European electricity markets are to remain unexplored as will be the issue
of insufficient transmission infrastructure, an essential facet of overall
supply security as well.\par


  \section{An Effective Solution To Defects Of Current Market Designs?}


  
First of all it should be mentioned that capacity instruments bring along their
own issues which merit attention when making a judgment about their
effectiveness in fixing the defects of existing market designs. Since planning
and construction of new power plants is a lengthy process, calls for bids in an
auction or contracts  in an administrative system should be launched far ahead
of actually drawing upon the additional reserves. Long-term supply arrangements
would also reduce investment risks for generators, which were identified as one
of the causes for the 'missing-money' problem. At the same time this shifts
risks like over-investment back to end-consumers\footnote{Hogan (2005), p.5} and
would also reduce competition among generators to offer the most cost-effective
solution to a lack of capacity\footnote{B\"{o}ckers, Giesing, Haucap, Heimeshoff
\& R\"{o}sch (2012a) p.81}.\par
  
  
  
    \paragraph{Incentive Structures \textemdash{} Plus \c{C}a change?}
\par
Current market designs are marred by a lack of price signals due to price
caps, be they inspired by regulatory or political motives. Policy makers and
regulators may desire low electricity prices so as to keep the economy
competitive in this respect in a globalized world. Or they may be concerned
about the affordability of electricity for certain parts of society and shield
them and business from the violent price swings a free market is prone to
induce. These curbs deny market mechanisms the ability to offer sufficient
incentives for private companies to invest in new power plants as they are
unlikely to recover their total costs. In order for capacity instruments to
provide these exact inducements they would at first have to come up with a
different way to determine the appropriate level of generation
capacity\footnote{Hogan (2005), p.5} if supply and demand cannot be balanced
through flexible prices. A single authority, be it the system operator or market
regulator, would have to decide upon the quantity of reserve power to contract
for or oblige distribution companies to do so.\par
   

Having arrived at an estimate about future load a price for capacity
needs to be established either by the contracting authority itself or via a
separate market where demand and supply for extra generation potential settle
on a specific price. The option of one institution deciding upon quantity and/or
price of reserve capacity raises the basic question, whether a centralized
decision process is able to accomplish this task in the first place. The same
information asymmetries persist as in an 'energy-only'--market, and it is hard
to imagine the responsible authority to correctly anticipate future demand and
plant availability all the time. In addition, such a system opens
opportunities for agency-capture by special interests, politicisation
or simply misjudgments about market conditions with potentially severe
consequences for the entire electricity market\footnote{Flinkerbusch \&
Scheffer (2013), p.24}.\par
 

How is a capacity market to discover the quantity of capacity necessary for
system stability? Price-based designs, see Battle \& Per\'{e}z-Arriaga (2008),
Diermann, von Hammerstein, Hermann, Matthes \& Schlemmermeier (2012) and
Siegmeier (2012) for details, also try to harness demand- and supply-side
valuations for finding a cost-effective solution to the 'missing-money' problem.
However, with regulatory price caps still in place capacity instruments will,
like traditional 'energy-only'--markets, fail to ensure enough generation power
is available at any time. Constraints on market prices applied to capacity
mechanisms will impede price signals to induce new investments like they do in
today's market designs.\par


To determine capacity requirements by administrative means and then to directly
contract private actors to meet them is a different approach to tackling
capacity shortages. But this would mean moving back closer towards the old model
of command \& control policy via publicly owned generation companies. Then
again, one of the initial aims of liberalization undertakings was to have
the market decide which projects to finance thus transferring investment risks
to the private sector. This objective would be undermined by a capacity
design that relied only on bureaucratic assessment.\par


 
 
\chapter{Conclusion}


Implementing a capacity mechanism expresses a mistrust of
'energy-only'--markets' ability to provide a satisfactory solution to the
'missing-money' problem. This paper has attempted to show why current market
designs fail to deliver adequate levels of electricity supplies during peak
loads, although some authors take a different view.\par


Hogan\footnote{Hogan (2005), p.8} for example argues that an
'energy-only'--market does not necessarily have to rely on spot-pricing alone
and that long-term supply contracts between generators and distribution
companies can be envisaged. B\"ockers et al. for their part judge the argument
that high electricity prices are politically unacceptable a false one because
short-sighted\footnote{B\"{o}ckers, Giesing, Haucap, Heimeshoff \& R\"{o}sch
(2012a) p.81}. Not the wholesale price is relevant they argue, but the total
costs consisting of capacity payments via surcharges or taxes added to
end-consumer prices. In the end these compound costs would equal peak prices in
a free market environment. Their reasoning however omits the political-economy
aspect of electricity market regulation. Since introducing complete cost
transparency to the retail level would look like price spikes to customers,
which are also voters, such a step will be less than appealing to politicians. A
detailed study of electricity markets' political-economy merits further research
but cannot be provided here.\par


The cornerstone of a capacity mechanism is the ability of its designer or
'master' to correctly assess current and future capacity requirements. If
capacity instruments are not allowed to use prices to arrive at this
estimation, they will fail to provide sufficient generation power like
'energy-only'--designs. As market incentives through price adjustments are ruled
out for political consideration the flaws of the existing market structure will
not be overcome.\par



%%%%%%%%%%%%%%%%%%%%%%%%%%%%%%%%%%%%%%%%%%%%%%%%%%%%%%%%%%%%%%%%%%%%%%%%%%%%%%%%


\chapter{Sources}

  \begin{itemize}
	
	\sffamily
	

	\item [\Rectsteel] \textbf{B\"{o}ckers, Veit, Giesing, Leonie, Haucap,
Justus, Heimeshoff, Ulrich \& R\"{o}sch, J\"{u}rgen (2012a):} Braucht
Deutschland einen Kapazit\"{a}tsmarkt f\"{u}r Kraftwerke? Eine Analyse des
deutschen Marktes f\"{u}r Stromerzeugung, in: \textsl{Ordnungspolitische
Perspektiven}, Issue 24, D\"{u}sseldorf.


	\item [\Rectsteel] \textbf{B\"{o}ckers, Veit, Giesing, Leonie, Haucap,
Justus, Heimeshoff, Ulrich \& R\"{o}sch, J\"{u}rgen (2012b):} Vor- \& Nachteile
alternativer Kapazit\"{a}tsmechanismen in Deutschland: Eine Untersuchung
alternativer Strommarktsysteme im Kontext europ\"{a}ischer Marktkonvergenz und
erneuerbarer Energien, \textsl{Study on behalf of RWE AG}, D\"{u}sseldorf.



	\item [\Rectsteel] \textbf{Creti, Anna\& Fabra, Natalia (2007):} Supply
Security And Short-Run Capacity Markets For Electricity, in: \textsl{Energy
Economics}, Vol.29, p.259-276.
	

	\item [\Rectsteel] \textbf{Diermann, Carsten, von Hammerstein,
Christian, Hermann, Hauke, Matthes, Felix Chr. \& Schlemmermeier, Ben (2012):}
Fokussierte Kapazit\"atsm\"arkte. Eine neues Marktdesign f\"ur den \"Ubergang zu
einem neuen Energiesystem, \textsl{Study on behalf of Umweltstiftung WWF
Deutschland}, Berlin.	


	\item[\Rectsteel] \textbf{Finon, Dominique \& Pignon, Virginie (2006):}
Electricity And Long Term Capacity Adequacy. The Quest Of Regulatory Mechanism
Compatible With Electricity Market, \textsl{Working Paper No.2}, Laboratoire
d'Analyse \'{e}conomique des R\'{e}seaux et Syst\`{e}mes energ\'{e}tiques,
Paris.
	

	\item[\Rectsteel] \textbf{Flinkerbusch, Kai \& Scheffer, Fabian
(2013):} Eine Bewertung verschiedener Kapazit\"{a}tsmechanismen f\"{u}r den
deutschen Strommarkt, in: \textsl{Zeitschrift f\"{u}r Energiewirtschaft},
Vol.37, p.13-25.


	\item[\Rectsteel] \textbf{Grave, Katharina, Lindenberger, Dietmar \&
Paulus, Moritz (2012):} A Method For Estimating Security Of Electricity Supply
From Intermittent Sources: Scenarios For Germany Until 2030, in: \textsl{Energy
Policy}, Vol.46, p.193-202.


	\item[\Rectsteel] \textbf{Hogan, William W. (2005):} On An
'Energy-Only' Electricity Market Design, Harvard University, Cambridge, MA
(USA).


	\item[\Rectsteel] \textbf{Hughes, William R. \& Parece, Andrew (2002):}
The Economics Of Price Spikes In Deregulated Power Markets, in: \textsl{The
Electricity Journal}, July, p.31-44.


	\item [\Rectsteel] \textbf{International Energy Agency (2002):} Security
Of Supply In Electricity Markets: Evidence And Policy Issues, Paris.


	\item [\Rectsteel] \textbf{International Energy Agency (2003):} Power
Generation Investment In Electricity Markets, Paris.


	\item [\Rectsteel] \textbf{Joskow, Paul L. (2008):} Capacity Payments In
Imperfect Electricity Markets: Need And Design, in: \textsl{Utilities Policy},
Vol.16, p.159-170.


	\item [\Rectsteel] \textbf{Joskow, Paul L. \& Tirole, Jean (2007):}
Reliable And Competitive Electricity Markets, in: \textsl{RAND Journal of
Economics}, Vol.38, p.60-84.


	\item [\Rectsteel] \textbf{Jurewitz, John L. (2002):} California's
Electricity Debacle: A Guided Tour?, in: \textsl{The Electricity Journal}, May,
p.10-29.


	\item [\Rectsteel] \textbf{Keller, Katja \& Wild, J\"org (2004):}
Long-Term Investment In Electricity: A Trade-Off Between Co-ordination And
Competition?, in: \textsl{Utilities Policy}, Vol.12, p.243-251.



	\item [\Rectsteel] \textbf{Marty, Fr\'{e}d\'{e}ric (2007):} La
S\'{e}curit\'{e} de l'approvisionnement \'{e}lectrique: Quels enjeux pour la
r\'{e}gulation?, in: \textsl{Revue de l'OFCE}, Vol.2007/2, Issue 101,
p.421-452. 

	
	
	\item [\Rectsteel] \textbf{Pollitt, Michael G. (2012):} The Role Of
Policy In Energy Transitions: Lessons From The Energy Liberalisation Era,
\textsl{Electricity Policy Research Group Working Paper No. 1208}, Cambridge,
Cambridgeshire (UK).


	\item[\Rectsteel] \textbf{Siegmeier, Jan (2011):}
Kapazit\"{a}tsinstrumente in einem von erneuerbaren Energien gepr\"{a}gten
Stromsystem, \textsl{Electricity Markets Working Papers No.45}, TU Dresden.
	
	
	\item [\Rectsteel] \textbf{Sioshansi, Fereidoon P. (2006a):} Electricity
Market Reform: What Has The Experience Taught Us So Far?, in:
\textsl{Utilities Policy}, Vol.24, p.63-75.
		
	
	\item [\Rectsteel] \textbf{Sioshansi, Fereidoon P. (2006b):} Electricity
Market Reform: What Have We Learned? What Have We Gained?, in: \textsl{The
Electricity Journal}, Vol.29, Issue 9, p.70-83.
	
	
	\item [\Rectsteel] \textbf{Woo, Chi-Keung (2001):} What Went Wrong In
California's Electricity Market?, in: \textsl{Energy}, Vol.26, p.747-758.	
	



  \end{itemize}







\end{document}