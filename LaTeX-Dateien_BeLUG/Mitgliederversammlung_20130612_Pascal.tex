% Created 2012-01-14 Sat 01:59
\documentclass[11pt,a4paper,ngerman]{article}
\renewcommand{\familydefault}{\sfdefault}
\usepackage[T1]{fontenc}
\usepackage[utf8]{inputenc}
\usepackage{fancyhdr}
\pagestyle{fancy}
\usepackage[ngerman]{babel}
\usepackage[unicode=true,pdfusetitle,
 bookmarks=true,bookmarksnumbered=false,bookmarksopen=false,
 breaklinks=true,pdfborder={0 0 0},backref=section,colorlinks=false]
 {hyperref}
\usepackage{breakurl}
\usepackage{lastpage}
\usepackage{textcomp}
\usepackage[official]{eurosym}
\usepackage{mdwlist}
%%%verkleinert den Abstand bei Listen und Aufzählungen zwischen denm einzelnen 
%%%Punkten
\usepackage{xcolor}


%%Definition von Grautoenen
\definecolor{dunkelgrau.80}{gray}{0.20}
\definecolor{hellgrau.60}{gray}{0.40}


% Kopf und Fusszeile
\renewcommand{\headrulewidth}{0.4pt}
\fancyhf{}

\lhead{Protokoll - Mitgliederversammlung 12. Juni 2013}
\rhead{\thepage \ | \pageref{LastPage}}
\cfoot{}


\title{Mitgliederversammlung - Protokoll}
\author{Berlin Linux User Group}
\date{12. Juni 2013}


%%%%%%%%%%%%%%%%%%%%%%%%%%%%%%%%%%%%%%%%%%%%%%%%%%%%%%%%%%%%%%%%%%%%%%%%%%%%%%%%


\begin{document}
\selectlanguage{ngerman}

\maketitle
\thispagestyle{empty}
\newpage


\setcounter{tocdepth}{2}
\tableofcontents

\newpage




\section{TOPIC I: Anwesenheit/Beschlussfähigkeit/Wahl des Versammlungsleiters 
und Protokollführers}

  \subsection{Anwesende Mitglieder}

    \begin{itemize*}
      \item Lutz Matscholl
      \item Bodo Eichstädt
      \item Klaus Montigel
      \item Norbert Ziese
      \item Andreas Gläser
      \item Pascal Bernhard
      \item Reinhard Peiler
      \item Friedrich W. Brockstedt
      \item Rainer Herrendörfer
      \item Frank Hildebrecht
      \item Sebastian Andres
      \item Ralf Vögtle
      \item Lutz Willek
      \item Gerhard Lüdtke
      \item Rüdiger Hanisch
      \item Christoph Koydl
      \item Claus Schäfer
      \item Arne Linus Eichstädt
      \item Philipp von den Linden
      \item Michael Rößler
      \item Thorsten Stöcker


   \end{itemize*}

    Anzahl: 20

  \subsection{Beschlussfähigkeit \& Wahl von Versammlungsleiter und 
	      Protokollführer}

	      Der Vorstandsvorsitzende stellte fest, dass die 
	      Mitgliederversammlung satzungsgemäß einberufen wurde und 
	      beschlussfähig ist. \\
	      Zum Versammlungsleiter wurde \emph{Friedrich W. Brockstedt} 
	      bestimmt. \\
	      Zum Protokollführer wurde \emph{Pascal Bernhard} bestimmt.



  \subsection{Genehmigung der Tagesordnung}

	      Es wurden keine Themen zur Tagesordnung hinzugefügt. \\
	      Die Tagesordnung wurde einstimmig angenommen.

\newpage{}	  
%%%%%%%%%%%%%%%%%%%%%%%%%%%%%%%%%%%%%%%%%%%%%%%%%%%%%%%%%%%%%%%%%%%%%%%%%%%%%%%%
	
	
\section{TOPIC II: Berichte zu Aktivitäten im 1. Halbjahr 2013}

  
  \subsection{Document Freedom Day [von: 
              \textcolor{hellgrau.60}{\textsl{Andreas Gläser, Lutz Matscholl}}]}

              
Den \textsl{Document Freedom Day 2013} können wir als insgesamt 
gelungen betrachten. Neben einem Vortrag von Andreas Gläser zum Thema 'Freie 
Dokumentformate' kam bei den Besuchern auch das gemeinsame Kochen und Essen gut 
an. Leider hatten wir keine Unterstützung durch die Free Software Foundation 
Europe (FSFE), was möglichweise daran lag, dass wir sie zu spät kontaktiert 
hatten.



  \subsection{LinuxTag 2013 [von: \textcolor{hellgrau.60}{\textsl{Ralf 
              Vögtle}}]}
	    
Dank sehr guter Personalorganisation und engagierten Mitgliedern können wir 
unsererseits mit der Unterstützung der Organisatoren des LinuxTags zufrieden 
sein. Kein einziger Helfer ist während der vier Messetage ausgefallen, auch war 
unser Stand stets mit einer angebrachten Anzahl von Mitgliedern besetzt. Der 
RepRap von Uwe (IN-Berlin) zog viele Messebesucher an unseren Stand und auch 
Haukes GPG-Programm trafen auf Interesse. Leider haben wir auch dieses Jahr 
wieder kein eigenes, überzeugendes Konzept der Eigendarstellung auf 
öffentlichen Veranstaltungen aufstellen können. Einerseits war der Stand für 
Messebesucher wenig attraktiv, sprich die Präsentation der BeLUG als Verein 
nicht zufriedenstellend. Zudem machte unser Stand einen unaufgeräumten Eindruch 
(beispielsweise standen die leeren Bierkisten nach der Standparty am ersten 
Messertag, die restlichen Tag weiterhin sichtbar am Stand herum). Die Anzahl 
der Brezeln bei der Standparty war leider nicht ausreichend.
  
  
  
  \subsection{Mitgliedsbeiträge [von: \textcolor{hellgrau.60}{\textsl{Reinhard 
              Peiler}}]}

Leider gibt es in der BeLUG weiterhin Mitglieder die ihre Beiträge nicht 
entrichten und auch nicht zu den Treffen erscheinen. (aktueller Stand: 17 
Mitglieder). Formalrechtlich sind zwei Mahnung zur Begleichung des 
Beitragsaustand erforderlich, bevor ein Mitglied aus diesem Grund aus dem 
Verein ausgeschlossen werden kann. Für den Ausschluss eines Mitglieds ist kein 
formeller Beschlus der Mitgliederversammlung erforderlich, da in der Satzung 
ein entsprechende Klausel zu finden. Somit wird der Vorstand diese Prozedur 
durchführen.


  \subsection{Bericht des Kassenwarts [von: 
\textcolor{hellgrau.60}{\textsl{Frank Hildebrecht}}]}

Gegenwärtig befinden sich auf dem Bankkonto der BeLUG 3842,98 EUR, in der Kasse 
??? EUR. In den nächsten Monaten wird das Konto von der Postbank zur 
Gemeinschaftsbank für Leihen und Schenken (kurz: GLS Bank) umgezogen. Bei den 
Zahlungsmodalitäten des Mitgliedsbeitrages wird es Änderungen zur 
bisherigen Praxis geben (Dies wurden per Mehrheitsbescheid von den Mitgliedern 
beschlossen). Die bargeldlose Entrichtung des Mitgliedsbeitrages wird weiterhin 
möglich bleiben. Der Beitrag muss, gleich ob per Überweisung oder mit Bargeld 
bis Ende Februar des betreffenden Jahres gezahlt werden. Neue Mitglieder müssen 
innerhalb von 4 Wochen der erfolgreichen Antragsstellung auf Aufnahme ihren 
anteiligen Beitrag entrichten.


  \subsection{Bericht zum Server [von: 
\textcolor{hellgrau.60}{\textsl{Lutz Willek}}]}

Bei \textsl{Thomas Krenn AG} wurde für 900 EUR ein neuer Server angeschaft, der 
mit einem Stromverbrauch von 145W deutlich sparsamer ist, als der Vorläufer. 
Hieraus ergeben sich geschätzt Kosteneinsparungen von ungefähr 200 EUR.


  \subsection{Bericht zur Küche [von: 
\textcolor{hellgrau.60}{\textsl{Lutz Matscholl}}]}

Leider mussten wir feststellen, dass der Abfluss in der Küche zum unzähligsten 
Male verstopft ist. Gerhard Lüdtke ist nach beträchtlichem Zeit- \& Geldeinsatz 
nicht mehr bereit, die Pumpe ein weiteres Mal zu reparieren. Jetziger Plan: 
zusammen mit dem IN-Berlin soll eine Tür oder ähnliches installiert werden, 
damit die Küche abgesperrt werden kann. Der Schlüssel hierfür soll nur 
kontrolliert herausgegeben werden.


  \subsection{Bericht zu Buchrezensionen [von: 
\textcolor{hellgrau.60}{\textsl{Claus Schäfer}}]}

Die Buchrezensionen sind für die BeLUG sehr einträglich, da sie für viele 
Besucher der Webseite der Einstieg sind, das heißt sie kommen auf der Suche 
nach einer speziellen Buchbesprechung zu uns. Leider stehen weiterhin drei 
Buchrezensionen aus, die noch abgeliefert werden müssen.


  \subsection{Bericht zu kommenden Vorträgen [von: 
\textcolor{hellgrau.60}{\textsl{Sebastian Andres}}]}

Für das zweite Halbjahr 2013 sind zwei Vorträge fest eingeplant mit den Themen: 
\textsl{Router} und \textsl{Solaris}.

%%%%%%%%%%%%%%%%%%%%%%%%%%%%%%%%%%%%%%%%%%%%%%%%%%%%%%%%%%%%%%%%%%%%%%%%%%%%%%%%

\section{TOPIC III: Geplante Aktivitäten im 2. Halbjahr 2013}


  \subsection{Helfergrillen}
  
Am 19. Juni wird das Helfergrillen in der Kulturfabrik stattfinden, um uns so 
bei den Helfern am LinuxTag für ihren Einsatz zu bedanken. Thorsten Stöcker 
bietet seine Hilfe beid der Vorbereitung des Helfergrillens an, obwohl er 
selbst nicht am LinuxTag präsent war.



  \subsection{Software Freedom Day}
  
Unser Engagement am Software Freedom Day (21.September) hängt von der 
Unterstützung der \emph{Free Software Foundation Europe} (FSFE) ab, das heißt 
ob ein Workshop angeboten wird oder nicht.


  \subsection{Linux Install Partys an Berliner Universitäten}

Dieses Jahr möchten wir neben der Unterstützung für Linux Install Partys an der 
Humboldt-Universität und der Technischen Universität auch der Linux-Gruppe 
SPLINE (Studentisches Projekt Linux Netzwerke) der Freien Universität unsere 
Hilfe anbieten.

%%%%%%%%%%%%%%%%%%%%%%%%%%%%%%%%%%%%%%%%%%%%%%%%%%%%%%%%%%%%%%%%%%%%%%%%%%%%%%%%

\section{TOPIC IV: Projekte}


  \subsection{Projekt: Raspberry Pi als Multimedia-Zentrale}
  
Für dieses Projekt ist die Anschaffung eines \emph{Raspebeery Pis} samt 
Speicherkarte und Gehäuse für insgesamt ca. 60 EUR geplant. Das Ergebnis soll 
auf dem großen Fernseher gezeigt werden. Ein Workshop ist angedacht.


  \subsection{Projekt: eLAB}
  
Das Projekt \emph{eLAB} befindet sich derzeit noch im Aufbau, bisher treffen 
sich die Interessenten dienstags und freitags. Handicap ist die fehlende 
Institutionalisierung des Projekts, das \emph{eLAB} ist (noch) kein Verein.


  \subsection{Projekt: Spacenet}

Ein Radioserver soll eingerichtet werden, der allen Besuchern von Hackerspaces 
weltweit einfachen Internetzugang mit nur geringem Aufwand bereitstellt. Die 
Identifikation erfolgt per Emailadresse. Dank gespendeter Hardware sind die 
Voraussetzungen für die Arbeit an diesem Projekt gegeben. Jedoch habe Sebastian 
Andres und Ralf Vögtle aufgrund ihrer Übernahme des Projektes 
\emph{Druckerserver} keine Zeit mehr für das Projekt \emph{Spacenet} und suchen 
hierfür Nachfolger.


\subsection{Themenabende}






\end{document}