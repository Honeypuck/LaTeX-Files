\documentclass[11pt,slidestop,compress,sans,handout,blackandwhite]{beamer}

\usepackage{beamerthemesplit}
\usepackage[ansinew]{inputenc}
\usepackage[ngerman]{babel}

%%Pakete f�r mathematische Symbole und Umgebungen
\usepackage{amsmath,amsfonts,amssymb}

%%Paket f�r Bilder
\usepackage{epsfig}




\title{Paketverwaltung unter Ubuntu}
\author{Pascal Bernhard (BeLUG)}
\date{\today}

\begin{document}
%% Folie 1 (Titelfolie)
\frame{\titlepage}

\section[Abschnitt]{}
\frame{\tableofcontents}

\section{Softwareverwaltung unter Linux}


%--------------------------------------------------------------------------------------%
%%%Folie 2

\subsection{1.Das Konzept Paketverwaltung}

\frame
{
  \frametitle{Was bedeutet Paketmanagement?}

  \begin{itemize}
  \item<1-> Linux-Software ist in Paketen organisiert
  \vspace{2cm}
  \item<2-> Softwareinstallation erfolgt �ber Repositories
  \vspace{2cm}
  \item<3-> Pakete sind teilweise voneinander abh�ngig      
  \end{itemize}
}


%--------------------------------------------------------------------------------------%
%%%Folie 3


\subsection{2.Ubuntu Software-Center}

\frame
{
  \frametitle{Wie installiere ich Programme grafisch?}

  \begin{itemize}
  \item<1-> Ubuntu Software-Center
  \item<2-> Softwareinstallation zentral �ber Repositories
  \item<3-> Pakete sind teilweise voneinander abh�ngig      
  \end{itemize}
}


%--------------------------------------------------------------------------------------%
%%%Folie 4


\subsection{3.Repositories}

\frame
{
	\frametitle{Was sind Repositories? Wo kommen die Pakete her?}
	
	\begin{itemize}
	\item<1-> Repositories sind Distributions-spezifisch
	\item<2-> Konfiguration der Paketquellen in \texttt{/etc/apt/sources.list}
	\item<3-> Hinzuf�gen von Repositories	
	\end{itemize}
}	
	
%--------------------------------------------------------------------------------------%
%%%Folie 5	
	
\subsection{4.Updates}

%--------------------------------------------------------------------------------------%
%%%Folie 6

\subsection{5.Installation/Entfernen von Paketen}

%--------------------------------------------------------------------------------------%
%%%Folie 7

\subsection{6.Sonderrepositories unter Ubuntu - PPAs}

%--------------------------------------------------------------------------------------%
%%%Folie 8

\subsection{7.Manuelle Installation von Paketen}







\end{document}
