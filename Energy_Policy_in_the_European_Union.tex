\documentclass[11pt,a4paper,english]{scrreprt}
\usepackage[T1]{fontenc}
\usepackage{lmodern}
\usepackage[ansinew]{inputenc}


\usepackage{babel}
%Option svgnames bietet mehr Farben
\usepackage[svgnames]{xcolor}
%\usepackage{xcolor}
\usepackage{relsize}
\usepackage{paralist}
\usepackage{typearea}
\usepackage{setspace}
\usepackage{textcomp}
\usepackage{marvosym}
\usepackage{amsmath}
\usepackage{array}
\usepackage{booktabs}
%\usepackage{hyperref}
\usepackage{url}

%Paket fuer farbige Tabellen
\usepackage{colortbl}


%%%----------------------------------------------------------------------
%%Definition von Grautoenennen
\definecolor{dunkelgrau}{rgb}{0.82,0.82,0.82}
\definecolor{hellgrau}{rgb}{0.92,0.92,0.92}

\definecolor{dunkelgrau.80}{gray}{0.20}
\definecolor{hellgrau.60}{gray}{0.40}

\onehalfspacing
\typearea[current]{calc}

\newcommand{\changefont}[3]{
\fontfamily{#1} \fontseries{#2} \fontshape{#3} \selectfont}

\newcommand{\textemph}{\textsl{\textbf}}
%%%----------------------------------------------------------------------
\begin{document}

\begin{spacing}{1}

	\begin{titlepage}
		
		\begin{center}
		{\large Freie Universit\"{a}t Berlin\\
		Otto-Suhr\texttwelveudash
		Institut f\"{u}r Politikwissenschaft\\
		PK 15371/15381 \\
		\emph{Projektkurs}\\
		\textsmaller{\textbf{Lecturers:} Prof. Dr. Miranda Schreurs \&
Dr. Helmut Weidner}}\par
		
		\vspace{4cm}
		
\changefont{ppl}{b}{n}
\sffamily{
  {\Huge{\textsmaller{\textbf{Term Paper}}}\\
 \bigskip	 
 \textcolor{dunkelgrau.80}{EU Energy Policy: Security Of Natural Gas Supply}}}
		
\par

\vspace{4cm}
		
\changefont{lmr}{m}{n}
		
		{\large Pascal Bernhard\\
		Schwalbacher Stra{\ss}e 7\\
		12161 Berlin\\
		Matrikelnummer: 3753179\\
		pascal.bernhard@belug.de}
		
		
		
		\end{center}

	\end{titlepage}
	
\end{spacing}

%%%----------------------------------------------------------------------


%%%----------------------------------------------------------------------
\tableofcontents


%%%----------------------------------------------------------------------
\setlength{\parindent}{30pt}
\setlength{\parskip}{0pt}


%%%-----------------------------------------------------------------------

\chapter{Introduction}


The European Union (EU) started out originally in 1951 as the
\textsl{European Coal and Steel Community}\footnote{The \emph{European Coal and
Steel Community} was founded by the Treaty of Paris on April 18th, 1951 and
entered into force on July 23rd, 1952} in an effort to jointly manage
the strategic goods coal and stell. In the following decades though, the area
of energy policy remained largely the preserve of its member states, not
even the oil crises of 1967 and 1973 induced any major shift towards integrating
energy policy thus making binding decisions on the European
level\footnote{Haghighi (2008), p.467}. Only with the publication of the first
Green Book on energy by the European Commission in 2000\footnote{see European
Commission (2000)} did the EU\footnote{Although strictly speaking this paper
will the pillar \emph{European Community} as its core subject, I will refer to
it as the European Union (EU). Regarding daily usage in the media readers will
be more familiar with this term, even though it is somewhat incorrect.}
forcefully express its desire to become active in this field. Subsequent
papers and statements reaffirmed this demand for increased competences for Union
institutions.\par

This lag in the integration process in the area of energy compared to other
policy fields contrasts markedly with the general development of the European
Community. The \emph{Neo-Functionalist} approach to international relations had
predicted that due to so-called 'spill-over effects' energy policy should become
more Europeanized, i.e. intergrated \textemdash{} the consequences of the common
market would require a common energy policy, distinct national strategies
rendered un uitable to an ever more integrated Europe\footnote{see Haas (1958),
(1964) as well as Lindberg \& Scheingold (1970)}. Until the 2000s there was no
European e ergy policy to speak of, although the internal market with its
provisions for the free movement of goods, labor, capital and services had been 
established by the Treaty of Maastricht already back in 1992. The ensuing
'spill-over effects' should then have prodded national governments to delegate
more and more powers to supranational institutions to better cope with
Europe-wide issues relating to energy.\par 


This \emph{Neo-Functionalists}' perspective can be questioned by energy's
essential role for modern, industrialized economies. Put simply without
sufficient energy supplies Europe's economies will quickly collapse, as
they are highly dependent on imports of primary
energy\footnote{\textcolor{dunkelgrau.80}{\textsl{The term 'primary energy'
describes an energy form found in nature which has not run through any
conversion or transformation process. It is contained in raw fuels, like crude
oil, natural gas, uranium and other forms of energy, e.g. wind, solar power
received as input to a system.}}} notably fossil fuels\footnote{see Eurostat
2012}. Large amounts of oil need to be purchased from foreign suppliers due to a
lack of domestic reserves. The same holds for natural gas, which is used for
heating and electricity generation. Being key to economic development and
national security energy policy fits neatly into \emph{Neo-Realist} and
\emph{Intergovernmentalist} explanations of international politics. Following
their arguments states should consider energy supply as a vital interest not to
be left to supranational organisations like the European Union. Rather member
states would keep energy policy firmly under their authority and delegate as
little power as possible.\par

Up to mid-2000s energy policy in Europe could be explained quite well from such
perspective with national governments preserving their prerogatives equally
with vigour as in foreign policy matters. Then came along the recurring
disputes between Russia, Europe's main supplier of natural gas on the one side,
and on the other the main transit country for these gas shipments, Ukraine. From
2005 on both parties argued constantly about the price to be paid for
natural gas and Russia's state-controlled gas company
\textsl{Gazprom}\footnote{\textcolor{dunkelgrau.80}{\textsl{This
paper will assume that the interests of Gazprom and the Russian state are
basically identical (the Russian state being the largest share-holder of the
company with 50.01\textdiscount{} of the shares and its chief executive
appointed by the Kremlin). Or at least they overlap to such a degree regarding
the export strategy for natural gas, that for the anaylsis presented here no
distinction will be made. For a more sophisticated view and the correctness of
this assumption please see for example Rosner (2006)}}} interrupted deliveries
several times or threatened to do so\footnote{The Telegraph (2008), Chicago
Tribune (2009)}.\par

For the first time some of the old member states were affected by supply
disruptions, although much less so than the new members from Central \& Eastern
Europe. As the disputes over gas became a rather permanent feature in
Ukrainian-Russian relations the European Commission became started repeated
initiative to arrive at a common energy strategy to better cope with such risks
to the natural gas deliveries and member states too began making efforts
towards enhancing their security of energy supplies. If these moves and
their results already constitute a policy shift that increased the level of
integraton in energy matter will be looked at in the following sections.\par





%%%%%%%%%%%%%%%%%%%%%%%%%%%%%%%%%%%%%%%%%%%%%%%%%%%%%%%%%%%%%%%%%%%%%%%%%%%%%%%%

\chapter{Research Question \& Structure Of The Paper}


      \paragraph{Research Question}

This paper looks for an answer to the question, whether the incremental
integration process, with multilateral cooperation in Europe evolving
from its first tentative steps after World War II to a monetary if not quite
political union, by now also includes to the area of energy policy. Put in
theoretical terms, which formal approaches to international relations and
specifically to the phenomenon of regional integrations do offer the best
explanations of this process? From the 'traditional' \emph{Neo-Realist}
perspective we should see no or very little integration, i.e. delegation of
powers to European institutions and joint decision-making by Union members. As
energy is tightly connected to national security concerns this school of
international relations predicts that states are anxious to hold on to as much
control as possible over choices on the energy mix and strategies to ensure
supply from foreign sources.\par

At the same time, the European Union has been delegated ever more powers with
members relinquishing fundamental attributes of their sovereignty in areas
such as monetary policy, regulation of their market economies and trade policy.
This paper does not intend to provide a detailed discussion on why governments
hand over such competences to a supranational body, even though they would
never do so from a (\emph{Neo-Realist}) national-security
standpoint\footnote{\textcolor{dunkelgrau.80}{In fact \emph{Neo-Realism}
argues that states enter only into international agreements which enhance their
national security. Economic motives are discarded by this approach as a valid
explanation of interstate cooperation.}}. There already exists extensive
literature on the European integration process in general and its
drivers\footnote{For a more detailed discussion of different theoretical
approaches to European integration see among others: Manners (2013), Pollack
(2001), Warleigh-Lack (2006)} for my work to offer genuinely new insights to the
discussion. Nonetheless research on the specific area of energy policy is rather
limited compared to other fields like trade, competition and environmental
regulation\footnote{Kelstrup 1998, p.18}. Here I will propose some explanations
of why the Europeanization\footnote{\textcolor{dunkelgrau.80}{In this paper the
terms \emph{integration} and \emph{Europeanization} will be used synonymously as
the integration process is exclusively happening in the European context}}
process for energy policy is evidently different and harder to understand by
existing integration approaches.\par

As already mentioned in the introductory part players of the EU have become
noticably active in energy matters only during the latter half of the past
decade. This raises the question of why no substantial energy policy had been
developed in the decades before, but suddenly we see the topic of 'energy
security' appear on the European agenda and various actors calling for a common
approach when dealing with foreign supplier countries. Do the recent
developments in EU energy policy call for adapting the theoretic approaches
political science uses when analysing this area of the integration process?
What causes can be identified for this change of direction in content and
structure of Europe's energy policy?\par


 




	\paragraph{Proceeding}

First this paper briefly presents the two main approaches to explaining
European integration and describe to what extent they are able to account for
the current situation of European energy policy. To provide the reader with
background information on the present-day picture of energy affairs in Europe
the workings of natural gas markets will be presented in more detail and its
implications for the concept of 'supply security'. 'Security of natural gas
supplies' will be the subject of a third part and I will limit my research
to this form of energy, leaving other energy types aside.\par

In order to show that there indeed is a policy change over time, the repeated
disputes between Russia and Ukraine about gas deliveries shall serve as
explanatory factors for the evolution of European energy policy in the last
decade. Here we should see responses from the EU that increasingly signal a
common energy policy rather than single-handed efforts by individual member
states when confronted with a threat of supply disruptions.\par

Drawing upon the work of Hermann and Gurr on crisis theory by J\"anicke I
suggest that it was these shocks in the form of recurring disruptions in gas
flows which induced a policy shift in EU politics that had been not possible
before. Expressed in theoretical terms the crucial variable 'adequate supply
of energy' change dramatically and suddenly. Here the paper will attempt to
show that member states and EU institutions realized they needed to react
swiftly with a response the existing institutional setting could not
provide. Thus the unexpected willingness on all sides to adapt the political
system via further integration steps.\par


	
	
%%%%%%%%%%%%%%%%%%%%%%%%%%%%%%%%%%%%%%%%%%%%%%%%%%%%%%%%%%%%%%%%%%%%%%%%%%%%%%%%


\chapter{Explanations Of European Integration}	
	

In the endeavor to offer a theoretical framework for explaining political
developments in Europe after World War II, two schools of thought,
\emph{Neo-Functionlism} and \emph{Intergovernmentalism} and its later, modified
variant \emph{Liberal Intergovernmentalism} have markedly shaped the debate on
regional integration processes among scholars of international
relations\footnote{Obydenkova (2011), p.3}. The phenomenon of European countries
relinquishing sovereignty and delegating power to a supranational organization
which started as the \textsl{European Coal and Steel Community} and then evolved
into the \textsl{European Community} which later became part of the
\textsl{European Union}, could not be understood by the standard approach at the
time \emph{Neo-Realism}.\par


      \paragraph{Neo-Realism}

This theory of cooperation or non-cooperation among nation states sees the
international system as intrinsically anarchic and structured by the
differing capabilities states possess\footnote{Waltz (1979), p.132/133}. States,
which are the only relevant actors, value military power and resources as
determinants of their respective capabilities and judge their own position on
the international stage relative to others. Consequently they are only concerned
with mere survival, and being uncertain about the true intentions of other
countries internatinal politics is marked by general mistrust. Thus states
only enter into cooperation with other countries when this serves the purpose
of national security, i.e. they can expect gains relative to the capability of
other states. Smaller states may have no other choice than to attempt to
constraint bigger powers through international rules in order to counterbalance
their superiority\footnote{Pollack (2001), p.224}.\par


   \section{Neo-Functionalism}

Following \emph{Neo-Realism}'s view of world politics Europeans should never
have entered into any arrangement which made them give up competences to
supranational institutions like the European Commission, the European Parliament
and the European Court of Justice. Put differently, such a thing as the European
Union which does not serve collective defensive purposes and thus national
security is unthinkable in Neo-Realist terms and cannot be explained.\par

\emph{Functionalism} proposed a new approach to understand the European
integration process. Its central idea was that a mismatch between the
territorial dimension of society's problems and of political authority to deal
with them induces pressures for jurisdictional reforms\footnote{Obydenkova
(2011), p.3}, i.e. a modification of the polical system. The expected gains from
joint supranational actions would push nation-states towards cooperation beyond
the narrow field of collective security. Political scientists like Haas,
Lindberg, Scheingold and Schmitter\footnote{see Haas (1961) \& (1964), Lindberg
\& Scheingold (1970), Schmitter (1969)}, for example, developed this concept
further by arguing that integration in one specific policy field engenders
so-called \emph{'spill-over effects'} which in turn initiate further integration
steps in related areas. As incomplete integration undermines the effectiveness
of policy, pressure builds up to deepen and extent international cooperation to
new sectors\footnote{Haas (1961), p.369}. This new school of though came to
known as \emph{Neo-Functionalism}.\par

\pagebreak 

	\subsection{Factors Shaping Regional Integration}

Regional integration processes are explained by Neo-Functionalists via the
interaction of three factors: 
  
  \textcolor{dunkelgrau.80}{
  \begin{enumerate}
     \item growing economic interdependence between countries
     \item capacity of supranational organization to resolve disputes and
establish international legal regimes
     \item international market rules that take the place of national market
rules
    \end{enumerate}
    }

\vspace{0.8cm}

The supply with natural gas from foreign sources is or may be conditioned
by the three factors mentioned above. European countries are highly reliant
on imported fossil fuels, for gas the share of imports stands at 67
\textdiscount{}, with Russia providing 24 \textdiscount{} of total EU gas
consumption\footnote{Eurogas (2012), p.6 \texttwelveudash{} figures for
2011}. This dependence on foreign supplies is complicated by the reliance on
transit countries not interfering with deliveries of natural gas via pipelines
running through their territory (For gas supplies from Russia, the most
important transit countries are Ukraine, Belarus and Poland). Since supplier
nations depend on the revenues they receive via sales of natural gas to Europe
or transit fees, we have a situation of interdependence, the first factor
\emph{Neo-Functionalism} regards as essential for regional
integration\footnote{Goldthau (2012), p.67}.\par

The second and third factors, the capacity of international organizations to
resolve disputes between states and establish as well as enforce rules on the
international energy market are basically two main topics of analysis of the
present research question. Integration of energy policy describes the powers of
EU institutions to set the rules of the 'energy game' in Europe and constrain
member states in their behavior towards foreign suppliers and vis-\`{a}-vis each
other. Initiated by interdependence between single member states and foreign
suppliers, Europeanization and more specifically the growing importance of
supranational institutions in this field create a feed-back loop.\par

As mentioned above the first integration steps should act a drivers for further
Europeanization since their 'spill-over effects' will make the drawbacks and
inefficiencies of incomplete integration visible to national governments. I will
treat the later two \emph{Neo-Functionalist} factors primarily as eventual
results of the integration pressure caused by the first factor, using them as a
measure of the degree of Europeanization. Their role in driving the integration
process will be left aside due to the limited scope of this paper.\par
	
	


    \section{Liberal Intergovernmentalism}
	
	
Like \emph{Neo-Realism} the theory of \emph{Intergovernmentalism} approaches
the phenomenon of regional integration from the perspective of
nation-states\footnote{Obydenkova (2012), p.4}. States and their governments
are considered to be the primary actors in international politics and rational
in the pursuit of their interests. During major intergovernmental negotiations
results are neither driven by 'supranational policy entrepreneurs' like the
European Commission (\emph{Institutionalism}'s standpoint), the consequence of
'spill-over effects' (as \emph{Neo-Functionalism} maintains) or answers to
demands by powerful interest groups (argument made by
\emph{Liberalism})\footnote{Pollack (2001), p.226}.\par

Moravcsik, one of the leading proponents of \emph{(liberal)
Intergovernmentalism}, views EU negotiations as a cooperative game in which the
level of cooperation reflects patterns in the preferences of national
governments\footnote{Moravcsik (1997), p.499}. The course of the integration
process is determined by interstate bargaining. Outcomes of intergovernmental
conferences result from a gradual convergence of national preferences towards
integration. Departing from \emph{Neo-Realist} assumptions liberal
Intergovernmentalists argue that these preferences about international politics
are generated on the domestic level and are not solely derived from national
security considerations\footnote{\textcolor{dunkelgrau.80}{This preference
formation on the domestic level with certain interest groups or individuals
involved gives this variant of \emph{Intergovernmentalism} the attribute
'liberal'.}}. Similarly, bargaining power is not only determined by military
strength and resources but rather by the intensity or salience of a country's
preferences\footnote{Moravcsik (1993), p.477}. Still Moravcsik admits that the
most powerful states in the EU determine the direction in intergovernmental
negotiations and treaties are concluded to their advantage\footnote{ibid
(1997), p.502}.\par

As to why nation-states would transfer competences to the supranational level
\emph{liberal Intergovernmentalism} offers an explanation somewhat analogous to
the one put forward by \emph{Functionalism}. The latter claimed that expected
gains from cooperative action would push national governments to joint
decision-making beyond questions of collective security. According to
\emph{liberal Intergovernmentalism} binding international agreements and
supranational institutions are attractive alternatives to either unilateral
action or non-binding cooperative accords that could be affected by
issue-linkage\footnote{\textcolor{dunkelgrau.80}{A negotiating strategy of
making agreement on one issue dependent on progress in negotiations about a
different issue. Combining multiple issues is used to change the balance of
interests in favor of a successful conclusion of an agreement.
\texttwelveudash{} Davis (2004), p.153}} during negotiations or the threat of
defection of one party shirking from its obligations\footnote{Moravcsik (1995),
p.612}.\par

In contrast to \emph{Neo-Functionalism}'s standpoint, the \emph{liberal
Intergovernmentalist} approach assigns supranational institutions only a minor
role in the Europeanization process itself. They are mere instruments to reduce
transaction costs in interstate bargaining by providing member states with
information. \emph{Intergovernmentalists} deny the existence of political
'spill-over effects' and the phenomenon of an organisation pushing for further
integration out of self-interest. In their perspective the European Commission,
for example, cannot be viewed as an 'activist bureaucracy' intent on being
delegated ever larger competences by member states\footnote{Pollack (2001),
p.227}.


%%%%%%%%%%%%%%%%%%%%%%%%%%%%%%%%%%%%%%%%%%%%%%%%%%%%%%%%%%%%%%%%%%%%%%%%%%%%%%%%

	
	
	
  \chapter{Energy Policy In The EU}
 
 
    \section{The Role Of Natural Gas For Europe's Energy Consumption}


The European Union used 400.0 MTOE\footnote{\textcolor{dunkelgrau.80}{The energy
unit \emph{Million Tonnes of Oil Equivalent (MTOE)} denotes the amount of energy
released by burning one tonne of crude oil. The precise calorific value of a
\emph{Tonne of Oil Equivalent} is defined by convention as different types of
crude oil have different calorific values. The IEA defines a TOE to have the
value of 41.868 GJ or 11.63 MWh (see IEA 2001). Applied to natural gas $1{}
m^{3}$ is the equivalent of $0.00093$ MTOE}} of natural gas in 2011. Out of a
total of 1704.1 MTOE of primary energy consumed\footnote{see Eurogas 2012} gas
thus has a 23 \textdiscount{} share in total primary energy use. This makes it
the secondest largest gas market in the world after North America. Natural gas
hasn't always played such a prominent role on European energy markets. Not until
the oil price shocks of 1973/1974 and 1979/1980 was the potential of gas
recognized to serve as a substitute for oil products\footnote{Institut f\"ur
Europ\"aische Politik (2013), p.8}. Three main uses may be distinguished for
natural gas:
  \begin{inparaenum}[(1)] \item fuel for heating space or water in the
residential and commercial sector, \item in industrial production gas is
utilized for heat generation, \item gas also serves as a fuel for electricity
generation even if it is of varying importance to member countries depending on
the energ mix in electricity production.\end{inparaenum} \par

Power generation is the most important use for natural gas, accounting for
roughly a third of the total amount consumed\footnote{ibid (2013), p.9}. New
technology like the efficient Combined Cycle Gas Turbine (CCGT) has made gas an
attractive fuel for electricity generation. The expansion of gas-powered
electricity capacity amounted to approximately 120 GW during the last decade,
surpassing installation increases in any other generation technology. Since gas
turbines emit less $CO_{2}$ than coal-fired power plants and are politically
less controversial than nuclear plants, current efforts to reduce carbon
emissions will probably see an increase their share in the energy mix. The
International Energy Agency (IEA) projects future demand for natural gas as a
primary energy source to rise as well its share in electricity genereation to
increase\footnote{see IEA (2012)}. The question how to ensure the gas shipments
do not get disrupted will thus become even more important in the future.\par


    \subsection{Disparities Among EU Member States}

The aggregate figures for the EU-27 as a whole conceal wide disparities in the
role natural gas plays for national consumption patters and the varying degree
of import dependency of EU countries. For the Netherlands the share of natural
gas is the highest in the EU with 44 \textdiscount{} of total primary energy
use, followed by Lithuania (36.5 \textdiscount{}), Romania (36 \textdiscount{}),
Italy (35.2 \textdiscount{}) and the United Kingdom (34.5
\textdiscount{})\footnote{Eurogas (2012), p.3 -- figures for 2011}. At the other
end of the spectrum we find Slovenia (11 \textdiscount{}), Finland (9.5
\textdiscount{}), Estonia (9 \textdiscount{}) and Sweden with a share of 2.5
\textdiscount{}. For comparison Germany, the largest member of the European
Union (in terms of population and size of its economy) uses gas for 20.5
\textdiscount{} of its primary energy demand and for France the figure is 14
\textdiscount{}.\par


	\paragraph{High Import Dependency}

As domestic production is insufficient to cover demand, a large proportion of
the gas consumed has to be imported. In 2010 the share of imports from foreign
suppliers was 62.9 \textdiscount, up from 43.5 \textdiscount{} in
1995\footnote{Eurostat (2012), p.20}. As a side note, the import dependency for
coal and oil stood at 39.4 \textdiscount{} and 84.3 \textdiscount{}
respectively. Algeria, Norway, Russia and Qatar are the main foreign suppliers
of natural gas for Europe, of which Russia is the most important with a share of
24 \textdiscount{} of Europe's total consumption or 37 \textdiscount{} of
imports\footnote{Eurogas (2012), p.6 -- figures for 2011}. Among the four
supplier countries mentioned only Norway can be considered as politically
reliable in so far as to have a stable, democratic political system and is
thus highly unlikely to use disruptions of gas shipments as a tool of foreign
policy as Russia has repeatedly been accused of\footnote{see Belkin, Nichol,
Ratner \& Woehrel (2013)}. \par

Taking a closer look at individual member states the picture of import
dependency also varies widely: \textcolor{dunkelgrau.80}{\emph{Please note that
the table on the next page shows reliance on Russian gas deliveries to satisfy
domestic demand, countries like Portugal and Spain are for their part completely
dependent on Algerian gas shipments. Since the papers focuses on supply issues
related to 'Russian' gas shipments only numbers on deliveries from Russia
are provided here.}}\par

%\vspace{0.5cm} 
\pagebreak

%%%serifenlose Schrift in Tabelle
\sffamily

  \begin{tabular}[l]{l c c}
    \rowcolor{black}
    \multicolumn{3}{l}{\textcolor{white}{\textbf{Share Of Russian Gas In Energy
Consumption}}} \\
    \rowcolor{black}
    \multicolumn{3}{l}{\textcolor{white}{\smaller{\textsl{(Source: BP
Statistical Review Of World Energy 2012, Eurostat 2012)}}}} \\

     {{\textcolor{Navy}{\textbf{Country}}}} & {{\textcolor{Navy}{\textbf{Natural
Gas}}}} & {{\textcolor{Navy}{\textbf{Primary Energy}}}} \\

    \hline




    \rowcolor{hellgrau} {\smaller{Austria}} & {\smaller{\textsl{14.2
\textdiscount{}}}} & {\smaller{\textsl{62.5 \textdiscount{}}}} \\

   {\smaller{Belgium}} & {\smaller{\textsl{0.0 \textdiscount{}}}} &
{\smaller{\textsl{0.0 \textdiscount{}}}} \\


    \rowcolor{hellgrau} {\smaller{Bulgaria}} & {\smaller{\textsl{10.8
\textdiscount{}}}} & {\smaller{\textsl{99.5 \textdiscount{}}}} \\
    
   {\smaller{Cyprus}} & {\smaller{\textsl{0.0 \textdiscount{}}}} &
{\smaller{\textsl{0.0 \textdiscount{}}}} \\


    \rowcolor{hellgrau} {\smaller{Czeck Republic}} & {\smaller{\textsl{18.4
\textdiscount{}}}} & {\smaller{\textsl{71.9 \textdiscount{}}}} \\

   {\smaller{Denmark}} & {\smaller{\textsl{0.0 \textdiscount{}}}} &
{\smaller{\textsl{0.0 \textdiscount{}}}} \\


    \rowcolor{hellgrau} {\smaller{Estonia}} & {\smaller{\textsl{6.0
\textdiscount{}}}} & {\smaller{\textsl{100.0 \textdiscount{}}}} \\

   {\smaller{Finland}} & {\smaller{\textsl{13.9 \textdiscount{}}}} &
{\smaller{\textsl{100.0 \textdiscount{}}}} \\


    \rowcolor{hellgrau} {\smaller{France}} & {\smaller{\textsl{2.9
\textdiscount{}}}} & {\smaller{\textsl{16.7 \textdiscount{}}}} \\

   {\smaller{Germany}} & {\smaller{\textsl{9.7 \textdiscount{}}}} &
{\smaller{\textsl{39.7 \textdiscount{}}}} \\


    \rowcolor{hellgrau} {\smaller{Greece}} & {\smaller{\textsl{5.7
\textdiscount{}}}} & {\smaller{\textsl{52.8 \textdiscount{}}}} \\

   {\smaller{Hungary}} & {\smaller{\textsl{24.9 \textdiscount{}}}} &
{\smaller{\textsl{65.0 \textdiscount{}}}} \\


    \rowcolor{hellgrau} {\smaller{Ireland}} & {\smaller{\textsl{0.0
\textdiscount{}}}} & {\smaller{\textsl{0.0 \textdiscount{}}}} \\

   {\smaller{Italy}} & {\smaller{\textsl{7.5 \textdiscount{}}}} &
{\smaller{\textsl{17.0 \textdiscount{}}}} \\


    \rowcolor{hellgrau} {\smaller{Latvia}} & {\smaller{\textsl{13.2
\textdiscount{}}}} & {\smaller{\textsl{100.0 \textdiscount{}}}} \\

   {\smaller{Lithuania}} & {\smaller{\textsl{38.8 \textdiscount{}}}} &
{\smaller{\textsl{100.0 \textdiscount{}}}} \\


    \rowcolor{hellgrau} {\smaller{Luxemburg}} & {\smaller{\textsl{0.0
\textdiscount{}}}} & {\smaller{\textsl{0.0 \textdiscount{}}}} \\

   {\smaller{Malta}} & {\smaller{\textsl{0.0 \textdiscount{}}}} &
{\smaller{\textsl{0.0 \textdiscount{}}}} \\


    \rowcolor{hellgrau} {\smaller{Netherlands}} & {\smaller{\textsl{3.6
\textdiscount{}}}} & {\smaller{\textsl{8.1 \textdiscount{}}}} \\

   {\smaller{Poland}} & {\smaller{\textsl{8.5 \textdiscount{}}}} &
{\smaller{\textsl{63.1 \textdiscount{}}}} \\


    \rowcolor{hellgrau} {\smaller{Portugal}} & {\smaller{\textsl{0.0
\textdiscount{}}}} & {\smaller{\textsl{0.0 \textdiscount{}}}} \\

   {\smaller{Romania}} & {\smaller{\textsl{5.6 \textdiscount{}}}} &
{\smaller{\textsl{17.0 \textdiscount{}}}} \\


    \rowcolor{hellgrau} {\smaller{Slovakia}} & {\smaller{\textsl{30.4
\textdiscount{}}}} & {\smaller{\textsl{98.2 \textdiscount{}}}} \\

   {\smaller{Slovenia}} & {\smaller{\textsl{6.3 \textdiscount{}}}} &
{\smaller{\textsl{56.2 \textdiscount{}}}} \\


    \rowcolor{hellgrau} {\smaller{Spain}} & {\smaller{\textsl{0.0
\textdiscount{}}}} & {\smaller{\textsl{0.0 \textdiscount{}}}} \\

   {\smaller{Sweden}} & {\smaller{\textsl{0.0 \textdiscount{}}}} &
{\smaller{\textsl{0.0 \textdiscount{}}}} \\


    \rowcolor{hellgrau} {\smaller{United Kingdom}} & {\smaller{\textsl{0.0
\textdiscount{}}}} & {\smaller{\textsl{0.0 \textdiscount{}}}} \\



    \hline
    \cellcolor{white} {\textcolor{Navy}{\smaller{\textbf{EU-27 Average}}}} &
{{\textcolor{Navy}{\smaller{\textsl{5.2 \textdiscount{}}}}}} &
{{\textcolor{Navy}{\smaller{\textsl{26.3 \textdiscount{}}}}}} \\


  \end{tabular}

\par

\newpage

%%%wieder Serifenschrift im Text
\rmfamily

Without generalizing too much one can say that member countries from Central \&
Eastern Europe are on average more dependent on gas deliveries from Russia and
thus more vulnerable to supply disruptions on this route. During the gas
conflicts between Ukraine and Russia in the years from 2005 to
2010\footnote{Reuters (2009)} it was them who were the first to be affected by
the interruption of gas flows\footnote{BBC (2006), Le Devoir (2009)}. These
variations in reliance on foreign supplies are a first indicator as to why
member states embrace different strategies to ensure reliable provisions with
natural gas with some preferring bileteral arragements\footnote{Umbach (2010),
p. 1231}. As a consequence a common European energy policy must look more or
less appealing to national governments depending on their specific supply
situation.\par


    
    
    
    \section{The International Gas Market}

As of today there is no international gas market where prices are determined on
a spot market by balancing supply and demand analogous to the working of the oil
market\footnote{Nies (2008), p.}. In fact one can distinguish three distinct
natural gas markets worldwide: East Asia, North America and Europe\footnote{CIEP
(2004), p.61}. Since most of the gas shipments pass through pipelines, producers
and consumers are tied to these transport facilities. Substantial sunk costs
characterise energy infrastructure, so long-term supply contracts are the norm,
often with a duration of 20 years or more\footnote{see BP (2012)}. Therefore
consumer countries cannot switch suppliers quickly, exposing them to risks
related to the producer side or the transit countries the shipments have to pass
through. Pipeline pose risks for producers as well, for they equally cannot
switch customers easily: pipelines are immobile infrastructure with starting and
end point predetermined for the duration of their life.\par

An alternative to pipelines as mode of transport is 'Liquefied Natural Gas'
(LNG). Unfortunately this solution requires considerable investment into
liquifaction and regazification plants, with costs reaching several billion
Euros. Europe has undertook considerable efforts over the past decade to
build up LNG capacities, so that about a quarter of natural gas imports now
arrives in its liquefied form\footnote{see Eurogas (2012)}. Unfortunately
81\textdiscount{} of LNG shipments comes from just three countries, Qatar
(47\textdiscount{}), Nigeria (18\textdiscount{}) \& Algeria (16\textdiscount{})
that do hardly fit the definition of a reliable supplier.\par




    \section{Security Of Supply}

	
	\paragraph{What Does 'Security Of Supply' mean?}


The questions where European countries get their natural gas from and whether
these provision can be relied upon in the short- and long-term are captured by
the concept of '\emph{Security of Supply}'. There exist several definitions of
'Security of Supply'\footnote{Winzer (2011), p.4}, but nevertheless they all
share the same idea of avoiding sudden changes in availability of energy
relative to demand. Although this paper will not delve into the nuances between
the various definitions, I shall present three different versions: Winzer
describes 'Supply Security' as "\textsl{the absence of, or protection from or
adaptibility to threats that are caused by or have an impact on the energy
supply chain}"\footnote{ibid (2011), p.9}. For the IEA "\textsl{Energy Security
is defined in terms of the physical availability of supplies to satisfy demand
at a given price}"\footnote{see IEA (2001)}. Stefanova on the other hand
explains the concept somewhat differently, but which may be more practical for
an analysis from a political science perspective: "\textsl{The concept of energy
security captures the entanglement of cost-benefit, reliability, affordability,
and sustainability concern in the provision of natural gas}"\footnote{Stefanova
(2012), p.52}.\par


	\paragraph{Unintentional Discontinuities}

Supply security can be jeopardized by different types of threats to the
constant flow of energy from producer to consumer country. The first
distinction which can be made here is between intentioned disruptions and
unintented discontinuities of natural gas supplies\footnote{Horsnell (2000),
p.11}. Natural catastrophes count among the unintended disruptions in case
transport infrastructure is so severely damaged as to put an effective halt to
further deliveries. The long-term effects of underinvestment in exploration and
development of new reserves or in transportation capacity will equally lead to a
shortfull in supplies. Arguably this can also be considered unintentioned in
nature for no single decision produces this outcome but rather it is the result
of a failed long-term investment strategy. Although the first threat to supply
security can occur suddenly, while the second makes itself felt over a much
larger time horizon, supply diversification equally provides a hedge against
both risks.\par



	\paragraph{Intentioned Supply Disruptions}

Following Horsnell\footnote{see ibid (2000)} we can imagine two scenarios
where deliberate actions by the supplier or consumer side cause an
interruption in gas deliveries. Consumer countries could impose an embargo on
energy exports of a specific country. This might be done to exert pressure on
the producer country's government for a policy change, as several states have
decided to do in order to induce Iran to forgo its ambitions of a nuclear
weapons program. The prospects of EU members blocking Russian gas imports, so
as to push Moscow to alter its policy towards its neighbours say, or allow more
press freedoms and free speech seems rather unlikely.\par

Finally gas supplies can be threatened in case the producer country either
decides to restrict its exports on short notice for political reasons or changes
its export strategy in general switching to different customers. Russia's
state-controlled gas company \emph{Gazprom} seems to have embarked lately on
such a long-run strategy shift with attempts to make deals with China on natural
gas exports to East Asia rather than its traditional customers in
Europe\footnote{Bloomberg BusinessWeek (2012), Energy Tribune (2011), EurActiv
(2012), Reuters (2012)}. For the longer term this could pose serious problems
for Europe as it would stengthen Gazprom's negotiating position, then being able
to choose among different customers. Having alternative export routes for its
natural gas to East Asia at the same time reduces Gazprom's and thus the Russian
state's dependence on revenues from sales to European consumers. In effect this
would change substantially the current situation of interdependence in which
Europe and Russia are mutually dependent on each other: the one side counting on
reliable gas supplies to satisfy its energy demand, the other relying on
predictable income through long-term supply contracts. As in the disruption
scenarios mentioned above a diversification strategy away from an
over-reliance on natural gas imports from one single source would increase the
European Union's security of energy supplies.\par 




	\subsection{Security Of Supply For Europe}



Sudden interruptions of provisions like the gas disputes between Russia and
Ukraine clearly put Europe's supply security into question. Several EU members
had been affected by diminished gas deliveries although they were not party to
the disagreement\footnote{European Commission (2009), p.4}. The cases of
interrupted gas supplies caused by disagreements between producer and transit
countries falls into the threat category of 'intentioned supply disruption'.
This paper does not have the room to discuss the underlying reasons for Russia
to halt its deliveries to Ukraine and thus eventually also to EU member states.
For the analysis at hand it suffices to state that this was a deliberate action
for whatever motive and can easilty happen again. The foremost interest of the
European Union is to reduce its vulnerability to such a supply shock, addressing
the fundamental causes of its occurence are relevant for Europe's long-term
energy strategy, immediate remedies have to look for different solutions.\par

	  \paragraph{Diversification}

The previous section referred to \emph{diversification} as a way to improve
supply security. 'Diversification' is a broad concept and encompasses several
steps to reduce one's vulnerability to sudden changes in the market
environment, the extreme case being a complete halt to energy deliveries. The
underlying idea is to avoid reliance on one energy source, type, technology or
means of transport to get the energy needed to end-customers\footnote{CIEP
(2004), p.2004}. First of all there is the possibility to receive one's energy
from a multitude of producer countries. As there would be several alternative
supply options available, problems with shipments from one particular source
should have limited impact on the overall supply situation. Diversification can
also be applied to the type of energy, i.e. finding appropriate substitutes for
the energy type in use (e.g. substituting oil for gas or vice versa). A
discontinuity in gas provisions could thus be balanced by using more oil.
Particularly relevant for natural gas is a diversification of supply routes,
since here cooperation from the producer is not sufficient to ensure supply
security but the transit country's collaboration is necessary as well.\par

Unfortunately the diversification potential for natural gas is considerably
lower than for other types of energy\footnote{CIEP (2004), p.67}. As burning oil
and coal for electricity generations emits significantly higher amounts of
$CO_{2}$ using these types of primary energy as substitutes for gas is
politically undesirable taking into account the EU's climate change targets.
In addition, existing gas pipelines may not be used for oil shipments. So the
options of diversifying away from natural gas as a primary energy source for
heating and electricity production are very restricted.\par

Diversifying supply routes also poses its challenges: pipelines are physically
fixed, thus either new pipelines will have to be built or a switch to different
transportation technology is needed. LNG terminals would provide EU
members the option to receive gas from other suppliers than their present ones.
Unfortunately, potential candidates Qatar, Nigeria, Egypt or Venezuela are not
very reliable as mentioned above and also bring supply risks with them. Even
though a consider number of LNG
reliquification plants\footnote{\textcolor{dunkelgrau.80}{Often overlooked in
the debate on security of natural gas supplies is the fact that seen globally
the number of liquification plants (Where natural gas is cooled down to about
-162 \textcelsius{} have to roughly equal the number of regasification
facilities. Thus newly built regasification plants in Europe may face
insufficient liquification capacity in producer countries.)}} have been
constructed in the EU over the last years, they are expensive to
build\footnote{Nies (2008), p.47} and thus beyond the means of smaller
members in Central \& Eastern Europe.\par




%%%%%%%%%%%%%%%%%%%%%%%%%%%%%%%%%%%%%%%%%%%%%%%%%%%%%%%%%%%%%%%%%%%%%%%%%%%%%%%%


\chapter{EU Energy Policy Up To The Gas Crises Of The 2000s}



As mentioned in the introductory part until quite recently the scope of
Europe's energy policy was rather limited when compared to other far more
integrated policy areas. Only since the adoption of the 'Lisbon Treaty' (Treaty
On The Functioning Of The European Community [TFEU]) on December 13, 2007,
was there a specific chapter on energy in the European
treaties\footnote{Braun (2011), p.1}.\par

The European Commission as the agenda setter for the EU legislative process was
very constrained in its possibilities to intitiate binding legislation that
would have obliged member states to adhere to common rules. It could only
attempt to do so by exercising so-called 'implicit powers', powers not granted
in the (numerously amended) foundation documents, to have joint measures adopted
for which there was no express treaty basis\footnote{Haghighi (2008), p.464}.
Specifically articles 95, 100 and 308 of the European Community Treaty [ECT]
(now the Treaty on the European Union [TEU]) provided the European Commission
with opportunities to expand the remit of EU law into new areas.\par
 
Still, measures related to energy questions had to be adopted by consensus when
submitted to the Council of Ministers\footnote{Art. 157 ECT}, giving
essentially every member country a veto against decisions perceived as harmful
to its interests. Here the Commission's effort to 'integrate' energy policy and
create an EU-wide market for energy alongside common rules for all have often
been rejected by national governments\footnote{Braun (2011), p.5, Umbach
(2010), p.1245}.\par

In its Green Papers on energy (2000 \& 2006) for example\footnote{European
Commission (2000), ibid (2006)} the European Commission had proposed the
creation of strategic gas reserves available to all EU members in case of a
crisis. The idea was dropped when it encountered considerable resistance in the
Council\footnote{CIEP (2004), p.68}. The benefits of such a costly measure would
have been unequally distributed since import dependence for natural gas
varies considerably among the different countries as shown in the section
\textbf{4.1.1}.\par

Because gas can only be stored in reserves given certain geological
conditions, which are not present in all European countries, the proposed
measure would also have required establishing an integrated European gas network
for the reserves to be effective in an emergency situation. That not only
involves sufficient investment funds but also mutual trust among members to act
in the common interest during a supply crises. This example highlights the
collective action problem posed by a European energy policy, when a joint
response is warranted but the cooperation of other members is still perceived as
uncertain. At the same time the issue of supply security did not seem to be
pressing enough for Europeans to transfer powers to Union institutions and 
establish oversight mechanisms to enforce compliance.\par


Generally speaking national governments have preferred to pursue individual
energy strategies, instead of pooling resources or even delegating competences
to EU institutions to control whether everyone adheres to the commonly agreed
strategy. Member states have backed their major energy firms by securing them
political support for business activities abroad on a bilateral basis,
especially when government are important\footnote{Pointvogel (2009), p.5709}.
The same is true for the internal market, where governments intervened in order
to delay liberalisation efforts by the Commission to shield their energy
companies from foreign competition.\par


%%%%%%%%%%%%%%%%%%%%%%%%%%%%%%%%%%%%%%%%%%%%%%%%%%%%%%%%%%%%%%%%%%%%%%%%%%%%%%%

\chapter{Ukrainian-Russian Disputes Over Natural Gas As An External Shock}

I will provide a very short chronology on the gas crises between 2005 and 2010
that lead to repeated interruptions of gas deliveries from Russia to Ukraine.
The news articles cited do not provide a complete picture of the events and
should be taken as such. For further information on the underlying causes of the
gas disputes and the political motives if the actors involved , if there were
any, the reader is advised to see the references.

    \section{Dispute 2005 -- 2006}


\begin{itemize}
	
\item Disagreement between Ukrainian gas firm \emph{Naftogaz} and
Russia's \emph{Gazprom} about gas supplies, prices, debt incurred by the
Ukrainian side\footnote{For a detailed account of the events see Stern (2006)}

  \begin{itemize}
   \item Gazprom demanded market prices for natural gas deliveries which had
been subsidized in the years since the breakup of the Soviet Union in 1992

   \item The Russians also claimed that Ukraine was siphoning off gas intended
for customers in the European Union\footnote{Gelb, Nichol \& Woehrel (2006),
p.2}. 

   \item No final agreement was reached during 2005.

  \end{itemize}


\item Russian gas shipments were cut off on January 1st, 2006


\item While Germany's and France's gas supplies remained virtually unaffected,
Poland saw its provisions from Ukraine/Russia decline by 14\textdiscount{} and
Hungary's imports were down by a quarter\footnote{BBC (2006)}

\item Then EU commissioner for energy Andris Piebalgs stated in a speech that
security of supply had become an important issue for the EU and that a
``\textsl{clearer and more collective and cohesive policy on energy supply}''
was needed\footnote{see Piebalgs (2006a)}.
	


	\end{itemize}


    \section{Dispute 2007 -- 2008}

\begin{itemize}


\item  On October 2nd, 2007 Gazprom threatened to interrupt gas supplies to
Ukraine if debt of \textdollar 1.3 billion for gas already delivered remained
unpaid.

\item On 5 January 2008 another warning was issued that if arrears of
\textdollar 1.5 billion were not settled the supply of natural gas would be
reduced.

\item At the end of February 2008 Gazprom threatened to halt deliveries from 3 
March on, in case gas shipments for 2008 were not paid in advance\footnote{BBC
(2008)}.

\item Supplies to Ukraine were reduced by 25\textdiscount{} on March 3rd and a
further 25\textdiscount{} the day after.



\end{itemize}


    \section{Dispute 2008 -- 2009}


\begin{itemize}

\item Several disputes between Ukrainian and Russian gas/transit companies
about prices and contract terms over the year 2008 saw Gazprom announcing in
December 2009 that failure to reach an agreement for supplies would result in
higher gas prices and eventually disruptions of supplies to Europe.

  \begin{itemize}
   \item Not only gas prices for 2009 were a point of contention, but also
outstanding debt by Naftogaz\footnote{Le Figaro (2009)}
  \end{itemize}

\item Starting from 4 January 2009 gas flows from Russia to Ukraine fell and
stopped completely on January 7th\footnote{Le Devoir (2009)}

\item On 8 January 2009 the Czech presidency of the European Council released a
statement in which both Russia and Ukraine were criticised for their lack of
will to resolve their differences. In addition, the declaration demanded that
both parties accept an independent mission made up of international experts
monitoring the flow of gas\footnote{see Czech Presidency of the Council of the
EU (2009)}.

\item After German chancellor Angela Merkel had talked with Russian prime
minister Vladimir Putin, the European Commission started negotiations with
representatives from Gazprom and Russian state officials to set up a monitoring
agreement including Ukraine.

\item All three parties, Russia, Ukraine and the European Union signed an
accord on an independent monitoring mission on 12 January,
2009\footnote{see Pirani, Stern \& Yafimava (2009)} 

\item The effects on gas supplies for EU member were considerable: Bulgaria,
Slovakia, Greece, Austria and the Czeck Republic experienced drops in gas flows
of more than 65\textdiscount{}. For the first time big member states were
affected by the cutoff (supply reductions in parentheses): Germany
(10\textdiscount{}), Italy (25\textdiscount{}), France (15\textdiscount{})

\end{itemize}





%%%%%%%%%%%%%%%%%%%%%%%%%%%%%%%%%%%%%%%%%%%%%%%%%%%%%%%%%%%%%%%%%%%%%%%%%%%%%%%%


\chapter{The Evolution Of EU Energy Policy}


The three crises between 2005 and 2009 which affected gas supplies to member
states to varying degrees offer a natural experiment on European energy policy.
As explained above there was no explicit energy strategy the EU pursued up to
this point in time, its institutions lacking the competences for doing so and
national governments unwilling to follow the Commission's ambitions to integrate
this policy field. From 2005 on, the European Union was recurringly confronted
with external supply shocks that the hitherto approache to supply security,
which can be qualified as rather \emph{Intergovernmentalist} had not faced
before.\par

In this section the paper will analyse which policy changes can be detected over
the course of the gas disputes between Russia and Ukraine. This will be
followed by a discussion on whether we can talk of structural evolution in
primary law, which calls for a rethinking on the kind of approach political
science should use when trying to explain the integration process in the area
of energy. As has been hinted at in section 5 on Europe's earlier energy policy,
arguably there have been noticeable modifications in the provisions of the
Lisbon Treaty of 2007.\par



  \paragraph{Extended Objectives Of EU Energy Strategy}

Before the Treaty of Lisbon for the European Commission to become involved in
energy policy had only been possible via the detour of initiating legislation in
the areas 'common market' and 'environmental regulation'. It thus had to invoke
'implicit powers' in articles 85, 100 and 308 of the ECT to justify actions on
the Community level\footnote{Haghighi (2008), p.465}. Starting with its Green
Paper 2000 \emph{Towards A European Strategy For The Security Of Energy Supply}
the Commission defined a new objective for EU energy policy: security of energy
supply. Since EU member are all, if to varying degrees, dependent on energy
imports from non-EU countries this new focus also implied a greater role for the
Union in foreign policy.\par



  \section{New European Activism}


	\paragraph{The Three Pillars Of Europe's Energy Strategy}

By then, at least according to the Commissioner Andris Piebalgs'
view\footnote{see Piebalgs (2006b)}, a common energy strategy was to be
constructed on three pillars: 
  \begin{inparaenum}[(1)] \item \emph{efficiency}, completion of the internal
market so as to reap the expected welfare gains; \item \emph{sustainability}:
fight against climate change, more efficient use of resources; \item
\emph{security of supply}. \end{inparaenum} 
The first two pillars evoked by the Commission for Energy fall into the classic
realm of EU competences: \emph{Common Market} and \emph{Environmental Policy}.
The third pillar was new and had become relevant only during the past decade
when security of natural gas supplies had turned out to be far from assured for
EU members\footnote{Winzer (2011), p.2}.\par


      \paragraph{The Commission's Initiatives}

Due to its limited competences in foreign energy policy as provided by the
Community treaties (see European Community Treaty \& Treaty of the European
Union), the European Commission used partnership agreements with neighbouring
countries to further its objective of enhancing supply
security\footnote{Prange-Gst\"ohl (2009),
p.5296}\footnote{\textcolor{dunkelgrau.80}{This is not to imply that energy
objectives were the only motives for the Commission to pursue these cooperation
agreements, or that they were the most important goals.}}. Already in 2003 the
Commission had expressed its intention to export internal market principals to
Europe's immediate neighbours and most important partners in the energy
sector\footnote{European Commission (2003), p.4}. Examples of the policy
frameworks used to develop a European energy strategy are:
  \begin{inparaitem}\item \emph{Black Sea Synergy} (2007)\footnote{ibid
(2007a)}; \item \emph{EU Strategy For Central Asia} (2007)\footnote{ibid
(2007b)}; \item \emph{Union For The Mediterranean} (2008)\footnote{ibid
(2008a)}; \item \emph{Eastern Partnership} (2008)\footnote{ibid (2008c)}. 
  \end{inparaitem}
The \emph{Energy Charter Treaty} between the European Union and countries in
the western Balkans is foremost example of this attempt to align neighbours'
interests to those of the Union\footnote{Prange-Gst\"ohl (2009), p.5299}.
Signed in 2005 the treaty's objective is the establishment of a common
energy market between all its signatories. Non-EU countries are obliged to
transpose the 'Acquis-Communautaire' for energy (internal market regulation \&
institutions) into national law. So far Europe's main supplier Russia for
natural gas has signed but then withdrawn from the Energy Charter Treaty,
considerably limiting the effectiveness of this approach to energy
security\footnote{Westphal (2011), p.2}.\par

Contrary to \emph{Intergovernmentalist} assumptions the European Commission in
this instance clearly takes on the role of an 'policy-entrepreneur'. Even if
some form of encouragement from national governments through informal channels
can not be excluded this supranational actors ostensibly aims to extend
his own role in formulating foreign energy policy. Driving foward partnership
agreements the European Commission may not have enlarged its actual powers in
the EU institutional setting, but at least it thus secured its agenda-setting
powers. \emph{Neo-Functionalism}'s factors for regional integration offer
far more convincing explanations for this integration development.\par


      \paragraph{Energy Action Plans}

Only one year after the first gas conflict in January 2006 the European
Council agreed in March 2007 on an ambitious comprehensive energy and climate
strategy, named \emph{Energy Action Plan}  for the years 2007--2009
(EAP)\footnote{see European Council (2007)}. In addition to giving energy
efficiency and energy conservation priority for Europe's future energy strategy
the EAP favours liberalized gas and electricity markets. More important for this
analysis the action plan calls for bolstered measures to ensure supply security
and for the first time defines a common approach to external energy
policy\footnote{Umbach (2010), p.1234}.\par


The next year the European Commission followed up the proposals made by the
European Council by issuing a '\emph{2nd Strategy Energy Review}' along with a
new '\emph{EU Energy Security and Solidarity Action Plan}'\footnote{see European
Commission (2008b)}. Here it identified specific issues and weaknesses in
existing EU policy that were obstacles to a truly integrated common energy
strategy. Among the areas suggested for coordination and EU-level action three
would have entailed either joint decision-making, thus relinquishing some
national sovereignty in energy matters, or pooling resources to improve the
supply situation of all member states: external relations with supplier
countries, infrastructure needs and establishment of strategic reserves as
potential crisis response mechanisms.\par


	\paragraph{Change Induced By A Systemic Crisis}

The crisis theory approach of J\"anicke and others on political change offers
some explanations as to how this sudden flurry of activity by Commission and
European Council has come about. The European Union fits the approach's notion
of a political system as a group of actors which interact continuously with
each other in more or less pre-determined structures\footnote{Hermann (1973),
p.45}. The interruptions of gas flows in the years 2005 to 2010 can be seen as
shocks that dramatically affected the critical variable of adequate energy
supply, without which modern economies like Europe's will collapse. The change
in this variable (diminishing flows of gas to certain EU members, calling into
question energy supply) was sudden and unexpected. Thus it had not been
anticipated by the political and economic system of the European Union. Since
the entire society is affected by such a crisis that disrupts its economic life,
the pressure on the political system to react to this dramatic change of the
critical variable is immense. Thus we have a situation in which the system has
to undergo either revolutionary change or modify its policy substantially to
remain in place\footnote{Gurr (1973), p.66}.\par

A clear-cut causality cannot be established here, but the EU-wide repercussions
of the gas conflicts seem to have demonstrated to political leaders in Europe
that a thorough rethink of the existing energy strategy was necessary. It is
difficult to find a different, equally plausible explanation for the abrupt
activity by EU institutions in the wake of the gas disputes where there had
been no initiatives in the past, let alone any substantial integration
progress. Equally the mere recognition that the question of supply security was
an issue no longer to be exclusively left to individual member states implies a
policy shift that points to the explanations proposed in this paper.\par



	\paragraph{Unilateral Moves By Member States}

The subsequent years did not see the implementation of the Commission's 2008
proposals\footnote{Umbach (2010), p.1236}. Rather some member states opted for
unilateral moves to garantuee gas flows in cases of Russian-Ukrainian
conflicts\footnote{Financial Times (2009)}. German, Italian, Dutch and French
gas companies, aided by political support from their national governments at the
time, have concluded deals with Gazprom to build new gas pipelines that
circumvent transit countries such as Ukraine and Poland. This would leave
shipments via this transport infrastructure unaffected by future disputes
between Russia and Ukraine that might see supply interruptions.\par

The 'Nord Stream' project, which is made up of Russian \emph{Gazprom}, German
\emph{E.ON} \& \emph{Wintershall}, Dutch \emph{Gasunie} and France's \emph{GDF
Suez} has already been built and runs from Russia to Germany under the Baltic
Sea, thus passing by the Baltic countries as well as Poland. The other project
'South Stream' with participation from Russia's \emph{Gazprom}, Italian
\emph{Eni}, in addition to French state-controlled electricity supplier
\emph{\'{E}l\'{e}ctricit\'{e} de France} and German \emph{Wintershall}, is still
under construction but to be completed in 2015. This pipeline will circumvent
Ukraine, connecting Southern Russia with Bulgaria, again increasing security of
gas supply for participating countries, but at the same time weakening the
position of other EU member in Central \& Eastern Europe that depend on
Ukraine-route to be reliable.\par




\par


  \section{Treaty Changes}


The Treaty of Lisbon or 'Treaty on the Functioning of the European Community'
(TFEU) saw a general shift from decision-making by unanimity to qualified
majority voting in the European Council. The same holds for many aspects of
energy policy. Article 194 TFEU provides that measures in the areas of energy
taxation, the choice of national energy mix, and regulating the exploitation of
energy resources are still to be decided by unanimity vote\footnote{Braun
(2011), p.2}. Member states also keep their right to conduct bilateral
energy relations with supplier countries as they see fit, although EU
competition regulation applies nevertheless to any bilateral agreement. The
European Parliament has been awarded an enhanced role, as its consent to the
conclusion of any international treaties by the EU with third parties (i.e.
supplier countries) is required\footnote{Art. 218 TFEU}.\par

The revised Article 17 TEU remains ambiguous on the European Commission's role
in external relations with third parties and the representation of a common
position in energy questions. It states that: ``\textsl{with the exception of
the common foreign and security policy [...], [the Commission] shall ensure the
Union's external representation}''. Splitting up policy making in the external
dimension of energy policy between the European Commission, the High
Representative For Foreign Affairs and the rotating Council presidency (with the
required consent of the European Parliament for international treaties)
appears to make foreign energy policy rather complicated\footnote{Braun (2011),
p.8}. At the same time the TEU declares that ``\textsl{the Union shall define
and pursue common policies and actions and shall work towards a high degree of
cooperation in all fields of international relations}''. The treaty
modification confirms the European Commission's agenda-setting powers but only
refers to its watchdog role as guardian of the Treaties, stopping short of
actually delegating competences to enforce a common energy policy on
non-compliant member states. In Braun's opinion the treaty changes
provide new channels for external representation in international relations with
third parties\footnote{ibid (2011), p.8}. Still, the important aspect of
solidarity among members in crisis situations is not mentioned leaving the
collective action dilemma of supply security unsolved.\par


%%%%%%%%%%%%%%%%%%%%%%%%%%%%%%%%%%%%%%%%%%%%%%%%%%%%%%%%%%%%%%%%%%%%%%%%%%%%%%%%
  
 
\chapter{Conclusion}
  
This paper set out to apply competing theories of international relations,
\emph{Neo-Realism}, \emph{liberal Intergovernmentalism} and
\emph{Neo-Functionalism} to the European integration process in the field of
energy policy. Whereas an \emph{Intergovernmentalist} approach fitted quite
well the years preceding the Ukrainian-Russian gas conflicts, it has
difficulties explaining the initiatives by the European Commission and
the European Council to reach a common energy strategy in the latter half of the
decade. Even taking into account the bilateral deals between Russia and some
member states or their major energy companies, \emph{Intergovernmentalism}
looses much of its explanatory power, as we should not see any changes in the
legal foundations of the EU (Treaty of Lisbon \& TEU) which push forward the
integration of energy policy. This shift stops short of being revolutionary
though, thus a definitive judgment on the current state of policy is rather
difficult to make.\par

\emph{Neo-Functionalist} explanations do not seem too convincing either, as it
is not clear why the 'spill-over effects' would take so long to make themselves
felt and induce further integration moves in related policy fields. They are a
rather weak predictor for future policy developments, if more than a decade has
to pass since the establishment of the Common Market in 1992 until 'spill-overs'
become effective enough to lead to more integration. For such a long time
period we can rule out other factors determining the integration process.\par
 
I argued that the changes in energy policy over time as described aboved
appear to have been caused by external shocks which lead to a systemic crisis
for the EU political system. The European Union was then forced to react to this
change in the crucial variable of adequate energy supply. Further research is
called for in order to determine which relative role these external shocks on
the one hand and long-term functionalist spill-over effects, which may be hard
to detect, exactly played in the policy shift, or whether a completely different
explanation is needed. The paper can only suggest correlations, providing an
unshakeable argument for causality in this regard is beyond its scope.\par


%%%%%%%%%%%%%%%%%%%%%%%%%%%%%%%%%%%%%%%%%%%%%%%%%%%%%%%%%%%%%%%%%%%%%%%%%%%%%%%%


\chapter{Sources}

  \begin{itemize}
	
	\sffamily
	
	
	\item [\Rectsteel] \textbf{Aleklett, Kjell, Jakobsson, Kristofer,
S\"oderbergh, Bengt (2010):} European Energy Security: An Analysis Of Future
Russian Natural Gas Production And Exports, in: \textsl{Energy Policy}, Vol.
38, p.7827-7843.
	


\item [\Rectsteel] \textbf{BBC (2006):} Ukraine gas row hits EU supplies 
(\textcolor{dunkelgrau.80}{http://news.bbc.co.uk/2\\
/hi/europe/4573572.stm} - accessed: 10/12/2012).	


\item [\Rectsteel] \textbf{BBC (2006):} EU nations have started to feel the
impact of Russia's axeing of gas supplies to Ukraine, as Moscow accused Kiev of
stealing EU supplies (\textcolor{dunkelgrau.80}{
http://newsvote.bbc.co.uk/mpapps/pagetools/print/news.bbc.co.uk/2/hi \\
/europe/4573 572.stm} - accessed: 10/12/2012).
	


\item [\Rectsteel] \textbf{BBC (2008):} Gazprom to reduce Ukraine's gas 
(\textcolor{dunkelgrau.80}{http://news.bbc.co.uk/2\\
/hi/business/7271604.stm} - accessed: 10/12/2012).



	\item [\Rectsteel] \textbf{Belkin, Paul, Nichol, Jim, Ratner,
Michael \& Woehrel, Steven (2013):} Europe's Energy Security: Options And
Challenges To Natural Gas Supply Diversification, \textsl{CRS Report for
Congress}, Congressional Research Service, Washington, D.C..


	
	\item [\Rectsteel] \textbf{Blom-Hansen, Jens (2008):} The Origins Of
The EU Comitology System: A Case Of Informal Agenda-Setting By The Commission,
in: \textsl{Journal of European Public Policy}, Vol. 15, Issue 2, p.208-226.



\item [\Rectsteel] \textbf{Bloomberg BusinessWeek (2012):} Gazprom Chases China
as Europe Demand Falters: Russia Overnight (\textcolor{dunkelgrau.80}{
http://www.businessweek.com/news/2012-02-20/gazprom-chases-china-as-europe-deman
d-falters-russia-overnight.html} - accessed: 10/31/2012).



	\item [\Rectsteel] \textbf{Boussena, Sadek (2010):} Les
d\'{e}veloppements sur les march\'{e}s gaziers et leurs cons\'{e}quences sur les
relations Russie-UE, \textsl{Cahier de recherche}, Issue 38, Laboratoire
d'\'{e}conomie de la production et de l'int\'{e}gration internationale,
Grenoble.



	\item [\Rectsteel] \textbf{BP (2012):} Statistical Review Of World
Energy 2012, London.



	\item [\Rectsteel] \textbf{Braun, Jan Frederick (2011):} EU Energy
Policy Under The Treaty Of Lisbon Rules: Between A New Policy And Business As
Usual, in: \textsl{Working Paper No.31}, Centre For European Policy Studies,
Brussels.



	\item [\Rectsteel] \textbf{Casier, Tom (2011):} The Rise Of Energy To
The Top Of The EU-Russia Agenda: From Interdependence To Dependence, in:
\textsl{Geopolitics}, Vol. 16, Issue 3. p.536-552



\item [\Rectsteel] \textbf{Chicago Tribune (2009):} Russia threatens to cut gas
to Ukraine (\textcolor{dunkelgrau.80}{http:// \\
articles.chicagotribune.com/2009-01-01/news/0812310777\_1\_gazprom \\
-officials-ukraine-and-russia-gas-shutdown} - accessed: 07/26/2012).



	\item [\Rectsteel] \textbf{Clingendael Institute For International
Relations [CIEP] (2004):} Study On Energy Supply Security And Geopolitics, 
\textsl{Report on behalf of DG Energy \& Transport}, The Hague.



	\item [\Rectsteel] \textbf{Correlj\'{e} , Aard \& van der Linde,
Coby (2006):} Energy Supply Security And Geopolitics: A European Perspective,
in: \textsl{Energy Policy}, Vol. 34, p.532-543.



	\item [\Rectsteel] \textbf{Czech Presidency of the Council of the EU
(2009):} Informal General Affairs Council Ended Today In Prague, \textsl{Press
Release}, 8 January 2009, Prague.

	

	\item [\Rectsteel] \textbf{Davis, Christina L. (1994):} International
Institutions And Issue Linkage: Building Support For International Trade
Liberalization, in: \textsl{American Political Science Review}, Vol. 98, Issue 1
p.153-169.



\item [\Rectsteel] \textbf{Le Devoir (2009):} Les coupures de gaz affectent
toute l'Europe (\textcolor{dunkelgrau.80}{http://www\\
.ledevoir.com/2009/01/07/226068.html} - accessed: 12/06/2012).

	

	\item [\Rectsteel] \textbf{Dobbins, Michael, Schneider, Gerald \&
Zimmer, Christine (2005):} The Contested Council: Conflict Dimensions Of An
Intergovernmental EU Institution, in: \textsl{Political Studies}, Vol. 53,
p.403-422.



\item [\Rectsteel] \textbf{Energy Tribune (2011):} The High Stakes Pipeline
Game (\textcolor{dunkelgrau.80}{
http://www.\\
energytribune.com/8883/the-high-stakes-pipeline-game} - accessed: 09/22\\
/2012).



\item [\Rectsteel] \textbf{EurActiv (2012):} Gazprom plans Vladivostok pipeline
to lessen reliance on Europe (\textcolor{dunkelgrau.80}{
http://www.euractiv.com/energy/gazprom-plans-vladivostok-pipeli-news-515761\_foo
tnote{?}utm\_source=EurActiv\textdiscount20Newsletter\&utm\_\\
campaign=f05a367d02-newsletter\_energy\&utm\_medium=email} - accessed:
10/31/2012).



	\item [\Rectsteel] \textbf{Eurogas (2007):} Long Term Outlook For Gas
Demand And Supply 2007--2030, \textsl{The European Union of the Natural Gas
Industry (Eurogas)}, Brussels.



	\item [\Rectsteel] \textbf{Eurogas (2012):} Statistical Report 2012,
\textsl{The European Union of the Natural Gas Industry (Eurogas)}, Brussels.



	\item [\Rectsteel] \textbf{European Commission (2000):} Green Paper:
Towards A European Strategy For The Security Of Energy Supply, \textsl{COM
(2000)}, 769 final, Brussels.



	\item [\Rectsteel] \textbf{European Commission (2003):} Communication
From The Commission To The Council And The European Parliament On The
Development Of Energy Policy For An Enlarged EU, Its Neighbours And Partners,
\textsl{COM (2003)}, 262/3 final, Brussels.



	\item [\Rectsteel] \textbf{European Commission (2006):} Green Paper:
A European Strategy For Sustainable, Competitive And Secure Energy, \textsl{COM
(2006)}, 317 final, Brussels.



	\item [\Rectsteel] \textbf{European Commission (2007a):} An Energy
Policy For Europe, \textsl{COM (2007)}, 1 final, Brussels.



	\item [\Rectsteel] \textbf{European Commission (2007b):} Black Sea
Synergy \textemdash{} A New Regional Cooperation Initiative, \textsl{COM
(2007)}, 160 final, Brussels.



	\item [\Rectsteel] \textbf{European Commission (2007c):} Regional
Strategy Paper For Assistence To Central Asia For The Period 2007--2010,
\textsl{COM (2007)}, 59 final, Brussels.



	\item [\Rectsteel] \textbf{European Commission (2008a):} Barcelona
Process, \textsl{COM (2008)}, 319 final, Brussels.



	\item [\Rectsteel] \textbf{European Commission (2008b):} Communication
From The Commission To The European Parliament, The Council, The European
Economic And Social Committee And The Committee Of The Regions - Second
Strategic Energy Review: An EU Energy Security And Solidarity Action Plan
{SEC(2008) 2870} {SEC(2008) 2871} {SEC(2008) 2872}, \textsl{COM (2008)}, 781
final, Brussels.



	\item [\Rectsteel] \textbf{European Commission (2008c):} Eastern
Partnership, \textsl{COM (2008)}, 823 final, Brussels.



	\item [\Rectsteel] \textbf{European Commission (2009):} The January
2009 Gas Supply Disruption To The EU: An Assessment, \textsl{SEC (2009)}, 977
final, Brussels.



	\item [\Rectsteel] \textbf{European Commission (2010):} Communication
On Energy 2020: A Strategy For Competitive, Sustainable And Secure Energy,
\textsl{COM (2010)}, 639/3, Brussels.



	\item [\Rectsteel] \textbf{European Council (2007):} Presidency
Conclusions March 8/9, \textsl{EC (2007)}, 7224/1/07 REV 1, CONCL 1, Brussels.



	\item [\Rectsteel] \textbf{European Union (2006):} Consolidated Version
Of The Treaty Establishing The European Community, in: \textsl{Official Journal
C}, 321 E/37, Brussels.



	\item [\Rectsteel] \textbf{European Union (2008):} Consolidated Version
of the Treaty On The Functioning Of The European Community, in: \textsl{Official
Journal C}, 115/47, Brussels.



	\item [\Rectsteel] \textbf{European Union (2010):} Consolidated Version
of the Treaty On European Union, in: \textsl{Official Journal C}, 83, Brussels.



	\item [\Rectsteel] \textbf{Epstein, David \& O'Halloran, Sharyn
(1994):} Administrative Procedures, Information, And Agency Discretion, in:
\textsl{American Journal of Political Science}, Vol. 38, Issue 3, p.697-722.



\item [\Rectsteel] \textbf{Le Figaro (2009):} Quand Poutine d\'{e}montre les
torts du pr\'{e}sident ukrainien
(\textcolor{dunkelgrau.80}{
http://www.lefigaro.fr/international/2009/01/10/01003-20090110ARTFIG\\
00201-quand - poutine-demontre-les-torts-du-president-ukrainien-.php} -
accessed: 12/06/2012).



\item [\Rectsteel] \textbf{Financial Times (2009):} Russia and Italy sign gas
supply deal (\textcolor{dunkelgrau.80}{http://www\\
.ft.com/intl/cms/s/0/afdd90d6-41a3-11de-bdb7-00144feabdc0.html} - \\
accessed: 01/05/2013).



	\item [\Rectsteel] \textbf{Franchino, Fabio (2000):} Control Of The
Commission's Executive Functions: Uncertainty, Conflict And Decision Rules,
in: \textsl{European Union Politics}, Vol. 1, Issue 1, p.63-92.



	\item [\Rectsteel] \textbf{Franchino, Fabio (2007):} The Powers Of The
Union: Delegation In The EU, Cambridge, UK.



	\item [\Rectsteel] \textbf{Geld, Bernard A., Nichol, Jim, \&
Woehrel, Steven (2006):} Russia's Cutoff Of Natural Gas To Ukraine: Context And
Implications, \textsl{CRS Report for Congress}, Congressional Research Service,
Washington, D.C..



	\item [\Rectsteel] \textbf{Goldthau, Andreas (2012):} A Public Policy
Perspective On Global Energy Security, in: \textsl{International Studies
Perspectives}, Vol. 13, Issue 1, p.65-84.



	\item [\Rectsteel] \textbf{Gurr, Ted R. (1973):} Vergleichende
Analyse von Krisen und Rebellionen, in: \textbf{J\"anicke, Martin
(ed.):} Herrschaft und Krise: Beitr\"age zur politikwissenschaftlichen
Krisenforschung, p. 64-89, Opladen.



	\item [\Rectsteel] \textbf{Haas, Ernst Bernard (1958):} The Uniting Of
Europe: Political, Social, And Economic Forces 1950-1957, London.



	\item [\Rectsteel] \textbf{Haas, Ernst Bernard (1961):} International
Integration: The European Union And The Universal Process, in:
\textsl{International Organization}, Vol.15, Issue 4, p.366-392.



	\item [\Rectsteel] \textbf{Haas, Ernst Bernard (1964):} Beyond The
Nation-State: Functionalism And International Organization, Stanford.



	\item [\Rectsteel] \textbf{Haghighi, Sanam S. (2008):} Energy Security
And The Division Of Competences Between The European Community And Its Member
States, in: \textsl{European Law Journal}, Vol. 14, Issue 4, p.461-482.



	\item [\Rectsteel] \textbf{Hermann, Charles F. (1973):} Indikatoren
internationaler politischer Krisen, in: \textbf{J\"anicke, Martin
(ed.):} Herrschaft und Krise: Beitr\"age zur politikwissenschaftlichen
Krisenforschung, p. 44-63, Opladen.


	\item [\Rectsteel] \textbf{Horsnell, Paul (2000):} The Probability of
Oil Market Disruption: With an Emphasis on the Middle East. Working Paper for
The James A. Baker III Institute for Public Policy, Houston.



	\item [\Rectsteel] \textbf{Institut f\"ur Europ\"aische Politik (2013):}
Europe's Energy Future \textemdash{} Natural Gas Supply Between Geopolitics And
The Markets: The Future Role Of Natural Gas - Trends And Projections For Demand
And Supply In 2030, \textsl{Study cofinanced by Statoil, Otto Wolff-Stiftung \&
The Royal Norwegian Ministry of Foreign Affairs}, Berlin.



	\item [\Rectsteel] \textbf{International Energy Agency (2001):}
Towards A Sustainable Energy Supply, Paris.



	\item [\Rectsteel] \textbf{International Energy Agency (2012):}
World Energy Outlook 2012, Paris.



	\item [\Rectsteel] \textbf{Kelstrup, Morten (1998):} Integration
Theories: History, Competing Approaches And New Perspectives, in:
\textbf{Wivel, Anders (ed.):} Explaining European Integration, p.15-55,
Copenhagen.



	\item [\Rectsteel] \textbf{Kiewiet, D. Roderick \& McCubbins, Mathew D.
(1991):} The Logic Of Delegation. Congressional Parties And The Appropriations
Process, Chicago.



	\item [\Rectsteel] \textbf{Kj\"arstad, Jan \& Johnsson, F. (2011):}
Prospects Of The European Gas Market, in: \textsl{Energy Policy}, Vol. 35,
p.869-888.



	\item [\Rectsteel] \textbf{Le Coq, Chlo\'{e} \& Paltseva, Elena (2011):}
Assessing Gas Transit Risks: Russia Vs. The EU, in: \textsl{Working
Paper No.12}, Stockholm Institute of Transition Economics, Stockholm.



	\item [\Rectsteel] \textbf{Lindberg, Leon N. \& Scheingold, Stuart Allen
(1970):} Europe's Would-Be Polity: Pattern Of Change In The European Community,
Englewood Cliffs.




	\item [\Rectsteel] \textbf{Manners, Ian (2013):} European Communion:
Political Theory Of European Union, in: \textsl{Journal of European Public
Policy}, Vol. 20, Issue 4, p.473-494.



	\item [\Rectsteel] \textbf{Mastenbroek, Ellen \& Veen, Tim (2008):}
Last Words On Delegation?: Examining 'The Powers Of The Union', in:
\textsl{European Union Politics}, Vol. 9, Issue 2, p.295-311.



	\item [\Rectsteel] \textbf{McCubbins, Mathew D. \& Page, Talbot
(1987):} A Theory Of Congressional Delegation, in: McCubbins, Mathew
D. \& Sullivan, T. (eds): \textsl{Congress: Structure And Policy}, p.409-425,
Cambridge, MA, United States of America.


	
	\item [\Rectsteel] \textbf{Moravcsik, Andrew (1993):} Preferences And
Power In The European Community: A Liberal Intergovernmentalist Approach, in:
\textsl{Journal of Common Market Studies}, Vol. 31, Issue 4, p.473-524.	


	
	\item [\Rectsteel] \textbf{Moravcsik, Andrew (1995):} Liberal
Intergovernmentalism And Integration: A Rejoinder, in:
\textsl{Journal of Common Market Studies}, Vol. 33, Issue 4, p.611-628.



	\item [\Rectsteel] \textbf{Moravcsik, Andrew (1997):} Taking
Preferences Seriously: A Liberal Theory Of International Politics, in:
\textsl{International Organization}, Vol. 51, Issue 4, p.513-553.



	\item [\Rectsteel] \textbf{Nies, Susanne (2008):} Oil And Gas Delivery
To Europe: An Overview Of Existing And Planned Infrastructure,
\textsl{\'{E}tude}, Institut fran\c{c}ais des relations internationales, Paris.


	\item [\Rectsteel] \textbf{Pick, Lisa (2012):} EU-Russia Energy
Relations: A Critical Analysis, in: \textsl{POLIS Journal}, Vol. 7, Summer 2012,
p.322-365.



	\item [\Rectsteel] \textbf{Piebalgs, Andris (2006a):} Speaking Notes 
Welcoming The Agreement Between Gazprom And Naftogaz, \textsl{SPEECH/06/01},
Brussels.



	\item [\Rectsteel] \textbf{Piebalgs, Andris (2006b):} External
Projection Of The EU Internal Energy Market, \textsl{SPEECH/06/712}, Brussels.



	\item [\Rectsteel] \textbf{Pirani, Simon, Stern, Jonathan \& Yafimava,
Katja (2009):} The Russo-Ukrainian Gas Dispute Of January 2009: A Comprehensive
Assessment, The Oxford Institute For Energy Studies, Cambridge, UK.



	\item [\Rectsteel] \textbf{Pointvogel, Andreas (2009):} Perceptions,
Realities, Concession: What Is Driving The Integration Of European Energy
Policies?, in: \textsl{Energy Policy}, Vol. 37, p.5704-5716.



	\item [\Rectsteel] \textbf{Pollack, Mark A. (1997):} Delegation, Agency,
And Agenda Setting In The European Community, in: \textsl{International
Organization}, Vol. 51, p.99-134.



	\item [\Rectsteel] \textbf{Pollack, Mark A. (2001):} International
Relations Theory And European Integration, in: \textsl{Journal of Common
Market Studies}, Vol. 39, Issue 2, p.221-244.



	\item [\Rectsteel] \textbf{Pollack, Mark A. (2003):} The Engines Of
European Integration: Delegation, Agency And Agenda Setting In The EU, Oxford,
UK.



	\item [\Rectsteel] \textbf{Prange-Gst\"ohl, Heiko (2009):} Enlarging
The EU's Internal Energy Market: Why Would Third Countries Accept EU Rule
Export?, in: \textsl{Energy Policy}, Vol. 37, p.5296-5303.



\item [\Rectsteel] \textbf{Reuters (2009):} FACTBOX - 18 countries affected by
Russia-Ukraine gas row (\textcolor{dunkelgrau.80}{
http://www.reuters.com/article/2009/01/07/uk-russia-ukraine-gas-\\
factbox-idUKTRE5 062Q520090107?sp=true} - accessed: 04/09/2013).



\item [\Rectsteel] \textbf{Reuters (2012):} Russia, China resolve oil pricing
dispute - paper (\textcolor{dunkelgrau.80}{http://\\
www.reuters.com/article/2012/02/28/russia-china-oil-idUSL5E8DS01V201\\
20228} - accessed: 09/11/2012).



	\item [\Rectsteel] \textbf{Rosner, Kevin (2006):} Gazprom And The
Russian State, London.



	\item [\Rectsteel] \textbf{Schmidt-Feltzmann, Anke (2011):} EU Member
States' Energy Relations With Russia: Conflicting Approaches To Securing
Natural Gas Supplies, in: \textsl{Geopolitics}, Vol. 16, Issue 3, p.574-599.



	\item [\Rectsteel] \textbf{Schmitter, Philippe C. (1969):}
Three Neofunctional Hypotheses About International Integration, in:
\textsl{International Organizations}, Vol.23, p.161-166.



	\item [\Rectsteel] \textbf{Stefanova, Boyka (2012):} European
Strategies For Energy Security In The Natural Gas Market, in: \textsl{Journal
of Strategic Security}, Vol. 5, Issue 3, p.51-68.



	\item [\Rectsteel] \textbf{Stern, Jonathan (2006):} The
Russian-Ukrainian Gas Crisis Of January 2006, The Oxford Institute For Energy
Studies, Cambridge, UK.



\item [\Rectsteel] \textbf{The Telegraph (2008):} Russia threatens to cut gas
supplies over Ukraine's unpaid bills (\textcolor{dunkelgrau.80}{
http://www.telegraph.co.uk/earth/energy/gas/3901414/Russia-threatens-to-cut-gas-
supplies-over-Ukraines-unpaid-bills.html} - accessed:\\
07/26/2012).



	\item [\Rectsteel] \textbf{Umbach, Frank (2010):} Global Energy
Security And The Implications For The EU, in: \textsl{Energy Policy}, Vol.
38, p.1229-1240.



	\item [\Rectsteel] \textbf{Waltz, Kenneth (1979):} Theory Of
International Politics, Reading, MA.



	\item [\Rectsteel] \textbf{Warleigh-Lack, Alex (2006):} Towards A
Conceptual Framework For Regionalisation: Bridging 'New Regionalism' And
'Integration Theory', in: \textsl{Review Of International Political Economy},
Vol. 13, Issue 4, p.750-771.



	\item [\Rectsteel] \textbf{Westphal, Kirsten (2011):} The Energy
Charter Treaty Revisited, \textsl{SWP Comments}, No.8, March 2011, Stiftung
Wissenschaft und Politik, Berlin.



	\item [\Rectsteel] \textbf{Winzer, Christian (2011):} Conceptualizing
Energy Security, \textsl{EPRG Working Paper No. 1123}, Electricity Policy
Research Group, University of Cambridge, Cambridge, UK.



  \end{itemize}




\end{document}